%---------导言文章格式设置-------
%-----宏包-----------
\usepackage{titletoc,titlesec,fancyhdr} %--版式--
\usepackage[left=1in,right=1in,top=1.25in,bottom=1in,headheight=16pt]{geometry} %-页面设置
\usepackage[colorlinks,linkcolor=black,anchorcolor=black,citecolor=black]{hyperref}%-超链接
\usepackage{xeCJK,fontspec} %--字体---
\usepackage[font=small,labelsep=quad]{caption}
\usepackage[font=small]{subcaption} %引用
\usepackage[super,square,numbers,sort&compress]{natbib}%--引用上标设置-
\usepackage{amsmath,amssymb,amsthm,esint,mathrsfs,amsfonts,cases,bm,array} %--数学环境----
\usepackage{booktabs,multirow,diagbox} %---三线表格--
\usepackage{graphicx,float,enumerate,multicol,color}%--图片等--
\usepackage{indentfirst,xpatch,setspace}%--行距缩进等
%-----------------------------------------------------
%-----------中英文字体设置----------------
\setmainfont{Times New Roman}
\setsansfont{Arial}
\setmonofont{Courier New}
\setCJKmainfont[BoldFont=方正小标宋_GBK,ItalicFont=方正楷体_GBK]{方正书宋_GBK}
\setCJKsansfont[BoldFont=方正大黑_GBK]{方正黑体_GBK}
\setCJKmonofont{方正仿宋_GBK}
\newCJKfontfamily\cukai{华文楷体}
\newCJKfontfamily \beiweikai{方正北魏楷书简体}
%\setCJKmainfont[ItalicFont={KaiTi}, BoldFont={SimHei}]{SimSun}
%\setCJKsansfont{KaiTi}
%\setCJKmonofont{FangSong}
%\setCJKmainfont[BoldFont={Adobe Heiti Std},
%    ItalicFont={Adobe Kaiti Std}]{Adobe Song Std}%
%\setCJKsansfont{Adobe Heiti Std}
%\setCJKmonofont{Adobe Fangsong Std}
%---------中英文设置结束-----------------
%----------------------格式定制-----------
%----------目录章节大标题拉近距离产生引导线--------------
\titlecontents{chapter}[1em]{\bfseries}{\contentslabel{1.5em}}
{\hspace*{-1.5em}}{\hspace{0.5em}\titlerule*[10pt]{.}\contentspage}
%--------正文之外的标题居中------
\titleformat{\chapter}{\centering\LARGE\bfseries}{\arabic{chapter}}{1em}{}
%---------修改章节定义,使章节页页眉格式改为fancy-----------
\xpatchcmd{\chapter}{\thispagestyle{plain}}{\thispagestyle{fancy}}{}{}
%-------章节标记定制为数字+名称---设置页眉页脚---------------
\pagestyle{fancy}
\renewcommand{\chaptermark}[1]{\markboth{\arabic{chapter}\quad #1}{}}
\fancyhf{}
\fancyhead[LO,RE]{\small~成教毕业设计报告~}
\fancyhead[RO]{\small~电子商务对国际贸易的促进及研究~}
\fancyhead[LE]{\small~\leftmark~} %leftmark为一级标记,rightmark为二级标记
\fancyfoot[LE,RO]{\small~\thepage~}
%--------格式定制结束--------------
%----------------------间距设置------------
%-------章节标题与上下文距离、行间距-----------------
\titlespacing*{\chapter}{0.5em}{0pt}{15pt}
\titlespacing*{\section}{0.5em}{10pt}{10pt}
\titlespacing*{\subsection}{0.5em}{6pt}{6pt}
%\onehalfspacing%---1.5倍行间距---
\setlength{\parindent}{2em}%----首行缩进距离---

%---------间距设置结束-------------
%------------------其他设置------
\bibliographystyle{unsrtnat}%--参考文献格式,按引用顺序排序---
%-----------页脚----------
\usepackage[norule]{footmisc}
\usepackage{scrextend}
\deffootnote{1.5em}{1em}{\makebox[1.8em][l]{注\hspace{-0.3pt}\thefootnotemark:~~}}
\deffootnotemark{\textsuperscript{注\hspace{-0.1pt}\thefootnotemark}}
%\renewcommand{\thefootnote}{注\hspace{-0.2pt}\arabic{footnote}}%-脚注----
\graphicspath{{figure/}} %--设置图片路径---
%-------------目录等名称改为中文--------------------------
\renewcommand{\contentsname}{目~~录}
\renewcommand{\bibname}{参考文献}
\renewcommand{\indexname}{索引}
\renewcommand{\figurename}{图}
\renewcommand{\tablename}{表}
\renewcommand{\appendixname}{附录}
\renewcommand{\proofname}{证明}
\renewcommand{\listfigurename}{图~目~录}
\renewcommand{\listtablename}{表~目~录}
%-------去除citenum在方括号中间的空白-----------
\makeatletter
\DeclareRobustCommand\citenum
   {\begingroup
     \NAT@swatrue\let\NAT@ctype\z@\NAT@parfalse\let\textsuperscript\relax
     \NAT@citexnum[][]}
\makeatother 
%--------------格式设置完毕--------------------