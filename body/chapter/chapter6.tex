\chapter{结论及展望}
\section{本文总结}
本文主要通过理论分析以及数值模拟的研究方法对高速飞行器周围流场的物理特性以及流场引起的气动光学效应进行了研究,得到结论如下:

(1)基于流体力学理论,在假设高速湍流流场满足N-S方程组以及完全气体状态理论的基础上,对大涡模拟及雷诺平均法进行了推演,并通过对雷诺应力建立计算模型完成流场的分析。基于$k-\varepsilon$模型及$k-\omega$模型的混合算法推导出了剪切应力传输(SST)模型,另外讨论了在高超声速情况下的流场运动特征,指出在高超声速情况下流体性质发生变化且不再满足上述假设,应考虑真实气体效应和气动加热等问题。

(2)设定了流体运动方程组数值计算前的定解条件,利用隐式迭代法对控制方程进行了离散,通过ICEM建立了三维导弹、机载光学凸台的模型,对近壁面流场的网格进行了加密优化处理。通过Fluent对不同运行速度下的流场进行了数值模拟,得到流场各项物理参数,并发现在超音速情况下剪切层随速度增加而越来越靠近飞行器壁面,同时基于完整的分析过程建立了流场数值模拟以及分析的具体流程。

(3)基于光的波动理论及线性光学理论分析了光束在非均匀流场中的传输特性,介绍了在时均流场中常用的光线追迹法以及推导了在湍流流场中的基于折射率脉动的统计光学研究方法。提出在气动光学效应的计算中可以在时间平均流场中引入网格模型以及在湍流部分引入亚格子尺度模型,讨论了混合后的非均匀流场的光学传递函数计算方法并且研究了偏折角、斯特列尔比等气动光学计算及评价方法。

(4)基于CFD的数值模拟结果,完成了密度到折射率的直接转换,讨论了高斯光束的传播特性及夫琅禾费衍射特征,利用Mathematica计算了在不同速度、不同入射角下光学窗口周围的光束波面畸变,发现沿前进方向射入光学窗口的光束波面畸变远远小于垂直于前进方向产生的波面畸变,并计算出不同入射角下运行速度为2Ma时光学孔径范围内的斯特列尔比在0.21至0.26之间浮动。采用光学传递函数计算得到了高斯光束经过混合流场前后的光斑变化,发现经过2Ma下凸台光学窗口处10cm厚的流场作用后,光束在像面上的光强不再成高斯分布,并且光束中心偏移量为(0.022cm,-0.013cm)。

本论文完成了不同速度下飞行器周围流场的数值模拟,将得到的流场结果转化为光学参数,基于光学传输的理论模型计算了光束经过高速湍流流场后的畸变情况,并对结果进行了分析,为进一步实际应用中的光学校正以及气动光学效应的削弱提供了有力基础。
\section{对未来工作的展望}

本文有待进一步的研究工作:

(1)通过风洞试验对数值模拟结果进行验证,同时增加对高超声速情况下的流场机理研究,考虑真实气体效应、气动加热等问题对光传输的影响。

(2)对气动光学的评价方法上进行改进,并且结合远场成像,完成对光信号的实际接收以及目标成像、跟踪。