\chapter{绪论}
\section{研究背景及意义}
%全球性的金融危机已经逐渐过去,在渡过最困难的时期之后,世界整体经济已经出现了起伏上升的复苏趋势,
随着全球经融危机的度过,世界经济已呈现上升趋势,中国在此过程中,成功地抓住机遇,发展自身,已经超越日本成为世界第二大经济体。
在此过程中,以互联网为基础的电子商务在经济发展中起了至关重要的作用。

电子商务主要通过完备的电子通信技术基于网络从事的商业往来,目前各界由于对电子商务的参与程度、理解方向不同,导致对电子商务的定义不明确,但可以确认的是,电子商务依托于信息网络技术,核心是物与物的交换,
操作主体可以看作为交易双方,事实上的主要途径是银行电子支付要点和电子结算要点,客户数据信息内容作为具体支撑点,形成新型的贸易模式。

在互联网高速发展的如今,全世界都已经离不开网络,也离不开电子商务,基于互联网络高速便捷的优点,买家和卖家能够在相距很远的地方甚至可以不用互相知晓地进行商业往来。目前,政府、学术界、商界等公认的电子商务模式可以主要归结为为:B2C,C2C,ABC,B2B,O2O等类型。
电子商务具备以下特点:
\begin{enumerate}[(1)]
\setlength{\itemsep}{0ex}
\item 合作性高,电子商务的整个流程可以看作是一个协调工作的过程,生产、运营、销售等都是不可或缺的环节,线上线下的支付、货物的运送等也都离不开银行、物流部门、通信中心、技术服务人员等相互默契的合作,协调完成,各环节都需要精密合作。
\item 通用性强,电子商务作为目前广受欢迎的经营方式,具有通用性,数字化的网络世界中的交互都是基于互联网的操作。
\item 高效快速,在电子商务环境下,互联网将整个地球纳入地球村,人们的工作已经不再受到时间、地点的限制,客户可以以非常简单的方式在短时间内完成以往复杂的业务活动,并且能够随时随地通过互联网查询交易、资金、等信息。
\item 便捷严谨,通过规范贸易规则,基于互联网平台发布交易模板,能够有效地规范交易流程、合约等,交易双方仅通过简单地填写模板就可以完成贸易,
而模板既可以简化交易,又可以排除人为地漏洞,提高系统的严谨性。
\item 安全手段多,任何时候,安全性是交易的重中之重,相比与传统手段,电子贸易由于是基于互联网的交易,
拥有各种安全系数高的保密手段,比如加密签名、权限控制等,但是同时由于互联网的高效便捷,一旦出现安全漏洞,损失也比较巨大,因此安全性需要作为电子贸易的第一考量。
\end{enumerate}

在国际贸易中,电子商务的重要性不言而喻,这是个大而范的概念,它随着互联网而崛起,发展速度之快令人咋舌。电子贸易的规模化出现在虚拟交易平台出现后,依靠计算机网络的发展,打破了地域、时间、市场规模的限制,通过快速的信息传播、交互,电子商务得以飞速发展,并且潜力巨大,打破了固有的传统交易模式,互联网的日新月异加速了电子贸易的发展。

电子商务依托网络,操作交易方便快捷,既安全但也存在风险,优点与缺点共存,目前看来它的有利之处更为突出,发展前景可观,因此对电子商务的研究非常有必要,本文主要从电子商务的利弊、发展方向等方面着手,浅谈电子商务的未来可能。

\section{wu}
一、电子商务特点分析

计算机要素和通信网络要素,和程序化商务流程以及标准化商务流程,加之配套系统和法律结构体系等共同组成了电子商务基本模式。此时我们可以理解为,电子商务运行,是将互联网作为基本运行基础,将交易双方视为操作主体,银行电子支付要点和电子结算要点实际上是主要途径,具体支撑点就是客户数据信息内容。随之形成崭新的商务模式,将其与老旧式交易模式进行对比发现,电子商务优点很多。首先是交易市场不受局限控制,国际贸易过程中,电子商务运行效率和质量不断提升,国际贸易空间和国际贸易场所,二者得到了前所未有的发展,使得国际贸易与贸易之间的距离以及时间等日渐缩减,此种模式之下,贸易程序和贸易过程,二者没有之前复杂,当前全球化特点和智能化特点以及简易化特点归电子商务所有,电子商务技术出现,长时间受到世界各行各业所青睐,企业整体均依赖于电子商务技术,进行全球商品供给、购买、输送,有了电子商务,便于企业运行和流程规划,一切都有了科学依据和操控支撑。虚拟交易平台的出现,主要依靠着计算机网络技术的发展,国际互联网应用促进了电子商务的不断向前发展,电子商务内在商务模式获取巨大发展空间和发展潜力,数字化技术所支持的互联网平台在一定程度上形成了一种虚拟空间,此种虚拟空间将电子商务作为活动主题。须知,网络是主要信息传递平台,与此同时,也是大量贸易活动得以正常进行的核心平台,作为消费者,可以在足不出户的情况下随心所欲购买商品,既方便又快捷。通过数次调查和分析可看出,本世纪初,电子商务贸易额在全球氛围内达到了近2亿美元,全球贸易总额中占有将近30%的市场份额,由此可见,电子商务应用发展速度较之前相比相当之快,而且电子商务效能得到了全方位、多角度的发挥。

二、电子商务对国际贸易的影响分析

国际贸易市场交易变化日新月异,虚拟市场在此种背景下生成,电子商务借助虚拟化运行方式来达成信息交换,促进市场空间开放性形式形成,并且此时的发展空间会愈来愈大,老旧式市场地域条件限制因素被剔除,信息网络安全就全球而言,市场广大,世界经济全球化的发展进程因此加快。应该了解到,信息流动所产生资本商品要素和技术要素,在全球范围内加速流动着,网络经济可谓是如火如荼的发展着,此种网络贸易环境下,不同国家和国家之间的经济贸易关系不断强化,合作机会日渐增加。另外,国际贸易经营主体变化趋势明显,虚拟公司出现就是主要代表,公司个体是现代信息通讯技术能够长足有效发展的核心渠道,公司个体会将本体核心技术显露出来,将更多的公司运用此种方式有效的衔接在一起,使其成为一个完美的整体,对于公司而言,提升市场能力势在必得,为市场提供商品和为市场提供服务才显得极其有价值可言。电子商务管理方式在国际贸易经营管理方式中的变化逐渐加大,交互方式网络运行机制对于电子商务而言,其变化程度极其深远,在此种发展背景之下,国际贸易中能够所展现出的最优化配置,借助国际贸易世界经济来完成,此时,若是从全球化角度进行分析,市场机制效能才完美彰显。老旧式贸易模式现在被此种贸易形势替代掉,四流一体成为主流,简而言之,就是将物流作为支撑,将资金流视为基本运作形势,商品流通是主体,创新整改后的经营管理模式特性正是如此,此种经营模式在实施商业贸易交流时将信息网络作为依据,并且站在不同角度之上实施真正意义上的互动。我们应该意识到,网络可以使生产者和用户以及消费者之间实现互动,供货制度和零库存生产,二者才能得以完美实现,商品流动性就会变得更加的符合市场规律,与社会需求相互契合,符合经济发展诉求,中间商就是我们所熟知的信息网络。

三、现存问题要点分析

电子商务若想得到长远发展与立足,应该借助信息基础建设来完成目标,但是现有信息基础设施建设还有待增强,并且此时的网络规模局限性十分明显。重要的网络安全问题没有透彻得到解决,网络对于电子商务而言,十分重要,网络数据传递、网络数据交换如此,安全性处理亦如是。商业交易在网络上进行,安全性未能得到透彻保障,当前我国经济出在转型时期,尚未形成较为成熟的市场,社会信用结构体系没有得到透彻完善,网络交易受影响因素主要涵盖了身份识别内容、商业机密内容和通讯安全内容,尤其是没有授权的信息非法篡改和信息内容恶意拦截,网络交易和客户信息记录保存等仍有些许漏洞。电子商务对老旧式商业法律和商业习惯造成冲击,老旧式商事立法和商事习惯中,有纸式模式成为主流,但是电子商务过度的去依赖网络空间,理解为不完全在纸张基础上进行交易,因此若是站在客观角度进行研究,老旧式立法办法和电子商务运作间从根本上讲没有联系。但是电子商务这一事项上所涉法律事务繁多,风险仍在,那么电子商务在具体应用阶段就会滋生诸多障碍因素。虽然当前网络信息和电子商务在我国有了法律条文规定,但权威性法律甚是匮乏,争端出现其实没有透彻解决的可能,这样一来,电子商务发展稳定性和电子商务发展快速性,二者都会受到影响。

四、解决办法分析

(一)顺势而为

全球经济一体化进程日渐加快的今天,国际贸易领域范围内,信息技术应用频率有所加快,世界上每个国家在发展商务的整个阶段内,电子商务都发挥着巨大效能,我国若想在本世纪竞争中独占鳌头,就必须将电子商务作为核心研究对象,只有这样,才能在一定程度上与发达国家“抗衡”且具备一同进步和发展的机会。

(二)依法而行

电脑空间法律实际上是世界性的,本世纪中国国际贸易是一个发展快速的阶段与过程,自身具备发展前景,电子商务此时就必须做出改变,国际贸易竞争也要向发达国家看齐,与国际立法去向靠拢,并立足于国际立法标准。电子商务发展,务必要与法律之间达成匹配,按照电子商务发展基本需求,立法部门应该强化电子提单管理力度和司法官权的强化力度,对合同法和票据法以及消费者权益保护法进行健全,在安全认证方面务必加大规范力度,在线支付与电子商务机构和企业管理方面也要重视起来,不同类型的电子商务贸易都要透彻规范和整改。

(三)由浅入深

基础建设要做到位,为电子商务发展提供强劲的物质基础支持,国家方面电子商务现在已经得到了大量影响,信息基础设施建设优良决定着机会和空间的发展效果,所以为了促进电子商务更好更优的发展,我们应该注重信息基础设施建设,与此同时,提升网络资源利用效率,长此以往,行业分割管理制度会愈加完善,体制日渐健全,资源效益会得到质的提升。中国经济发展,电子商务方面会受到过高收费限制。

五、结束语

今后国家贸易发展,电子商务在其中起到了至关重要的效能,外贸企业务必要坚持自身原则,提升竞争力才是根本,要对电子商务进行深度认知,建立站点,供求信息扩大,建立客户群和交流渠道,树立优质企业形象,在国家贸易中利用电子商务手段谋求更为广阔的天地。 



\section{国际贸易中电子商务的影响}

(一)电子商务的积极影响

自从在国际贸易运用了电子商务,国际贸易的复杂流程变得非常简便。相比于传统的国际贸易,如果国际贸易非常频繁,就会使得国际贸易间的手续流程非常繁杂,耽误了很多的时间,存在很多的误差之处,贸易效率低。而随着电子商务的使用,国际贸易立刻发生巨变,因为电子商务是一种无纸交易,把传统贸易模式下的复杂琐屑过程变得非常快捷,得到很多国家的重视。无纸交易是指在现代信息技术的帮助下,通过互联网的快速传播,使得双方的交易信息迅速传达至对方,这样一来就可以减少传统贸易间的订单需求等,快速的达成贸易目的。无纸交易的主要核心是电子数据间的交换,简称EDI,EDI是指利用计算机技术和互联网的方便特点简化国际贸易的业务流程。和传统贸易相比,不需要前期的贸易准备、中期的贸易技术管理人员洽谈和最后贸易订单签订和人员跟进等。因此使用电子商务可以大大减少订单的各种准备工作,提高贸易的效率,同时又除去了不必要的成本投入。

(二)电子商务的消极影响

电子商务带来方便的同时,不可避免也造成了一些消极的影响。首先电子商务加剧了发达国家与不发达国家之间的差距,因为电子商务使用的基础取决于高度发达的电子信息技术和互联网技术等先进的科技技术,而电子商务的快速使用又使得国家间的贸易业务变得更为广阔,加速了国际贸易的发展,提高了国际贸易的效率。因此在这种不公平的发展基础上,发达国家比不发达国家更有优势,发达国家与发展中国家的差距将会越来越大,出现马太效应。其次,相比较于传统的贸易模式,电子商务主导的贸易模式极易造成大量的税款流失。因为电子商务的快速发展,造成了大量贸易的无纸交易,虽然可以加快贸易的效率,但不可否认也在一定程度上影响了发展中国家的税收,导致发展中国家的发展滞后。由于互联网虚拟的特点,很容易使得无纸交易变成欺诈交易,造成国际贸易的混乱,影响国家的正常收税活动,同时由于电子商务的无纸交易,导致了国内中介的衰败,影响国家税收。 

 