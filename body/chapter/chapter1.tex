\chapter{绪论}
\section{研究背景及意义}
后金融危机时代,世界经济在渡过最困难的时期之后开始呈现起伏上升的复苏特征,但总体基础仍然较脆弱,复苏过程将缓慢曲折。在金融危机的冲击下,各国都在经济政策特别是与中国经济息息相关的外贸和能源政策上有所调整和改变,中国经济发展的外部环境危机重重。同时,中国经济的内生环境也出现了诸多新的不确定性因素:自然灾害频仍、房地产市场进入宏观调控期,人力资源成本上升,流动性下降;固有的经济结构上的深层次矛盾进一步地阻碍经济的和谐发展。经济复苏期,对中国经济来说,挑战与机遇并存。只有深刻认识内外部环境的新特点、新趋势,增强加快经济发展方式转变的自觉性和主动性,调整和优化经济发展战略才能不断在经济发展方式转变上取得新的实质性进展。

\section{经济复苏期中国经济发展的外部环境特征}

从目前情况分析,中国经济所面临的外部环境也将有所改善,国民经济总体形势企稳向好。根据国际货币基金组织对2011年世界经济的预测,全球经济增长为3.1$\%$,发达国家平均增长为1.3$\%$,新兴经济体和发展中国家平均增长将高达5.1$\%$。金融市场的稳定性调控方面,通过各个国家通力合作,金融机构房贷信心和市场融资功能得到了有效地恢复。通过采取政府主体投资、政府入股银行等等模式和手段,外贸企业流动性瓶颈得到了有效地缓解。尽管如此,可以看到,金融危机后的国际间社会经济因素、地缘政治因素相互交错,旧有的矛盾进一步深化,新的利益的博弈开始显现,促成中国对外经济发展的环境出现新的演化趋势:

1.主要经济体寻求自我发展倾向增加,体外贸易保护主义抬头。金融危机后全球经济缓慢走向复苏,各经济体经济外部化程度不一样因此受危机冲击的程度也不一样,从而导致缺乏国际间合作基础,经济体之间利益很难协调。在国内就业压力和政治力量博弈下,主要的经济体(国家)自主发展趋势明显,各国或经济体将解决自身体内就业、社会稳定、基础产业的发展等问题作为首要问题,陆续出台各种贸易限制措施,贸易保护主义逐步抬头。

2.中国传统优势产业领域进入者增加,国际市场竞争日趋激烈,出口面临更大压力。国际金融危机发生后,发达国家将重振制造业作为就业出路之一,不惜进入一些能效低、规模大、利润薄的低端领域,希望通过扩大出口规模来缓解国内就业率低迷和总体贸易赤字的问题。另外,许多发展中国家出口总体竞争力逐渐提高,面对日趋激烈的国际市场,有可能通过本币贬值等手段加速争夺低端产品领域的国际市场。未来,中国将在中高端产品领域面对发达国家的强有力的挑战,而在低端产业领域同时面对发展中国家和发达国家更加激烈的竞争。

3.能源价格可能再次攀升,国内企业经营风险增加。数据预测,2010年全球资源和能源产品的需求将有所回升,为应对金融危机,各国政府多采取为低利率等较为宽松的货币政策和积极的财政政策,从而导致投资需求增加和主要货币汇率走低,能源类大宗商品作为保值和投资的渠道之一,交易价格可能再次攀升。能源价格上涨将带动中国企业特别是制造业产品的生产成本,加大中国企业进出口的经营风险(霍达等,2009)。另外,随着其他发展中国家进入国际市场、大力发展国际贸易,中国出口产品的成本优势将被消弱,竞争日趋激烈,企业发展空间被进一步压缩。 技术性贸易壁垒成为全球经济力量博弈的重要手段。所谓技术性贸易壁垒,是指采取强制性或非强制性确定商品某些特征的技术法规或技术标准,还有旨在检验商品是否符合这些技术法规或技术标准的认证、审批或试验程序,从而形成事实上的贸易障碍。随着世界贸易市场自由化程度的提高,传统关税壁垒和数量限制措施在当前已经相对采取的较少,技术性贸易壁垒成为当前国际贸易领域重要的非关税壁垒。根据世贸组织的数据,从1995—2007年,各成员方向世贸组织通报的影响贸易的新规则总量达23 897件,这其中涉及技术性贸易壁垒的规则的事件达到16 794件,占总量的71$\%$。
 
\section{中国经济发展的内生环境的新特征}

从国内经济看,经济回升的基础还不稳定、不巩固、不平衡,一些固有的深层次矛盾特别是结构性矛盾仍然突出。同时,伴随全球经济复苏过程和中国经济的快速增长,国内经济发展环境新的问题和矛盾开始体现出来。

1.经济增长对政策拉动形成依赖,缺乏支撑经济增长的内生动力。数据显示,今年来,中国经济呈现出“增长靠投资、投资靠政府”的线性增长特征。政策拉动的外部效果就是透过政府的投资,国有资本开始大规模进入竞争性领域,从而挤出了社会投资。社会投资受到对市场信心降低、国外需求不足、融资门槛过高、相关市场准入性限制措施等影响无法实现有效地增长机制。体制环境内缺乏从政府投资向民间投资的增长动力转换趋势。研究甚至表明,投资过快增长对中国经济效率的提高产生的是抑制作用(吕冰洋、余丹林,2009)。另外,国内消费增长也同样是靠政策引导和补贴,在国民收入分配体制没有发生根本性改变的情况下,大多数国民可支配收入没有显着的改善,稳定的消费增长内生机制就很难形成。还要注意的是,在政策拉动的机制下,政府体系的投融资平台贷款将积累大量的系统性金融风险。政府出于融资和保障基础性设施项目的资金需求,建立了各种政府投融资平台,后者作为承贷主体统一向银行等金融机构申请贷款,之后再将贷款转贷给相关的企业或项目,使债务转而信贷化,债务风险隐藏于贷款中。近两年来的新增贷款的增量就大部分流向地方政府融资平台公司。 环境性约束持续刚性、自然灾害增加,已成为影响中国经济发展的重要因素。中国的生态环境基础原本就比较脆弱,在人口压力和粗放型经济等多种因素作用下,目前中国生态安全形势已十分严峻。加之,自然灾害频仍,经济增长的资源支撑能力下降,经济发展的未来空间收缩。2008年年初的低温雨雪冰冻造成贵州、湖南、江西等地的经济损失,汶川特大地震造成四川、甘肃、陕西等受灾地区的基础设施、农业和工业等损失严重。两次灾害的直接经济损失分别为1 516亿元和8 451亿元。研究表明,中国已经成为世界上自然灾害最严重的国家之一,每年由于气象灾害所造成的经济损失是2 000亿元~3 000亿元,约占GDP总量的3$\%$。 人力资源的成本上升,劳动力流动性下降。沿海发达地区产业转移和产业升级的不断加快、国家区域均衡发展政策的密集出台,促成中国经济发展的区域布局和产业布局格局发生根本性变化。作为劳动力传统输出大省,现在已经成为承接产业转移的主要目的地,本地劳动力的需求增加,与沿海地区劳动力密集型加工制造业和传统服务业对劳动力的需求形成竞争局面。2009年有7个省已经对最低工资标准作出了调整,其中最高档平均调整幅度达到17$\%$左右。还有20个省计划将在2010年适时调整最低工资标准。希望借助调整最低工资标准,增加对人力资源的吸引力。而从人力资源的供给来看,劳动力的流动性意愿呈现下降趋势。随着年龄的增加,老一代农民工选择返乡就业的比重明显增大。而“80后”、“90后”出生的所谓新生代农民工逐渐成为劳动力市场的主力。新生代农民工多是独生子女,较父辈所受教育水平更高,对工资回报和个人发展空间更为重视,就业选择往往集中在几个较发达的经济中心城市。今年年初出现的“用工荒”现象一个显着的特征就是,在沿海发达地区“用工荒”矛盾再度尖锐的同时内地企业招工也出现困难。

4.经济结构上的深层次矛盾进一步凸显。国内经济发展结构中的供需矛盾进一步深化,中国的高速经济增长依然与结构性问题相伴随(高帆,2010)。需求方面,国内消费增长长期处于启而不动、后劲不足的尴尬中。居民收入持续增长难度较大是影响消费后劲的根本性因素。数据来看,中国居民消费仅占GDP的36$\%$,处于世界主要经济体和国家中的最低水平。供给方面,产业产能过剩问题突出,产业结构调整的内容和复杂性增加。受金融危机和外贸需求降低的冲击,以制造出口为导向的产业链条的产能过剩问题尤为明显。数据显示,钢铁行业产能过剩近2亿吨,水泥行业产能过剩约5亿吨,铝化工、造船业、煤化工等等行业也均存在较突出的产能过剩问题。此外,受流动性过剩和投资意愿增加的拉动,各地新材料、新能源及相关制造设备等新兴产业项目密集上马,市场尚未培育成熟已经形成新的产能过剩问题。同时,产业产能过剩调整的具体实施还需要考虑到所在地的经济发展规划、附属企业的生存、区域就业和社会稳定等多种因素。

\section{国际贸易中电子商务的影响}

(一)电子商务的积极影响

自从在国际贸易运用了电子商务,国际贸易的复杂流程变得非常简便。相比于传统的国际贸易,如果国际贸易非常频繁,就会使得国际贸易间的手续流程非常繁杂,耽误了很多的时间,存在很多的误差之处,贸易效率低。而随着电子商务的使用,国际贸易立刻发生巨变,因为电子商务是一种无纸交易,把传统贸易模式下的复杂琐屑过程变得非常快捷,得到很多国家的重视。无纸交易是指在现代信息技术的帮助下,通过互联网的快速传播,使得双方的交易信息迅速传达至对方,这样一来就可以减少传统贸易间的订单需求等,快速的达成贸易目的。无纸交易的主要核心是电子数据间的交换,简称EDI,EDI是指利用计算机技术和互联网的方便特点简化国际贸易的业务流程。和传统贸易相比,不需要前期的贸易准备、中期的贸易技术管理人员洽谈和最后贸易订单签订和人员跟进等。因此使用电子商务可以大大减少订单的各种准备工作,提高贸易的效率,同时又除去了不必要的成本投入。

(二)电子商务的消极影响

电子商务带来方便的同时,不可避免也造成了一些消极的影响。首先电子商务加剧了发达国家与不发达国家之间的差距,因为电子商务使用的基础取决于高度发达的电子信息技术和互联网技术等先进的科技技术,而电子商务的快速使用又使得国家间的贸易业务变得更为广阔,加速了国际贸易的发展,提高了国际贸易的效率。因此在这种不公平的发展基础上,发达国家比不发达国家更有优势,发达国家与发展中国家的差距将会越来越大,出现马太效应。其次,相比较于传统的贸易模式,电子商务主导的贸易模式极易造成大量的税款流失。因为电子商务的快速发展,造成了大量贸易的无纸交易,虽然可以加快贸易的效率,但不可否认也在一定程度上影响了发展中国家的税收,导致发展中国家的发展滞后。由于互联网虚拟的特点,很容易使得无纸交易变成欺诈交易,造成国际贸易的混乱,影响国家的正常收税活动,同时由于电子商务的无纸交易,导致了国内中介的衰败,影响国家税收。 

\section{浅议动态国际贸易理论}
 1动态贸易模型和产品升级

发达国家研制出的新产品往往价格较高,不发达国家效仿出该产品后,在价格上往往能具有竞争优势。在发达国家,劳动力分为两种:从事生产的劳动力和从事研发的劳动力。在不发达国家的劳动力,或者从事传统行业,或者从事模拟和学习新产品的部门。不管是发达国家还是发展中国家,其利润都需要用无风险利率来贴现。首先来寻找这个动态模型的均衡稳定状态。在均衡的稳定状态,发达国家以速度为g进行创新,不发达国家以速度为u进行效仿。发达劳动力在研发和生产之间的分配比例是常数,发达国家的产品被发展中国家效仿的比例也是常数。每个代表性消费者最大其终身效用函数。在最优路径上,支出的增长满足欧拉方程。均衡状态下,发达国家和不发达国家的消费增长速度是相同的。在均衡状态下,支出和生产的增长速度相同。

2非均衡状态下的贸易演进

随着不发达国家经济的增长以及技术的进步,发达国家和不发达国家之间的贸易模式也会发生动态改变。不发达国家经济情况变化包括:劳动生产率的增长,劳动人口的增加,财富的增长等。研究表明,1990年至2005年,发达国家劳动生产率平均增长1.46$\%$,而不发达国家劳动生产率增长3.87$\%$。不发达国家劳动生产率近年来的增长率明显高于发达国家。可以证明:不发达国家劳动生产率增长越快,发达国家商品被不发达国家进行效仿和模拟的速度也会加快。不发达国家劳动生产率的提高会降低不发达国家的仿制成本。因此不发达国家会效仿生产更多的高精尖产品。不发达国家生产的产品价格便宜,因此会提高两国人民的购买能力。购买能力的提高也会导致对发达国家高精尖产品的需求,也提高了发达国家的创新动力。不发达国家劳动生产率的提高会缩小两国工资差距。我国人口劳动生产率近年来发展迅猛,这对我国贸易产品升级起到了重要作用。如何从贸易大国转变为贸易大国是我国目前面临的重要课题之一。不可否认,我国在很多产品的生产制造上已经位居世界领先水平,但和西方发达国家仍然存在差距。大力发展科学技术无疑是促进我国贸易产业结构升级的重要措施。以美国生产洗衣机为例,数据显示:从1980年至2008年,美国生产洗衣机的数量逐渐下降,首先是英国和韩国的生产数量上升,其次是新兴国家:巴西、俄罗斯、中国等。这种动态变化的原因是:技术的扩散以及发展中国家价格较低的生产要素。再比如:洗碗机。首先普及使用洗碗机的国家是美国,时间是1950年。然后洗碗机逐渐在英国、瑞典、法国。英国等国普及洗碗机的时间是1959年。在二十世纪六十年代,比利时、德国、奥地利等国家也先后使用了洗碗机。洗碗机普及比较晚的欧洲国家是葡萄牙和希腊。从洗碗机的普及时间差异中可以看出。通常是经济比较发达的国家,首先使用新产品。经济水平相对落后的国家在新产品的使用时间上也会相对较晚。电冰箱在不同欧洲国家的普及时间也呈现出类似的规律。另外,笔记本电脑、手机、VCR等产品,也呈现出类似特征。除了分析不同国家的新产品使用普及时间的差异,还可以类似分析我国不同省份电脑普及时间,从中也可以发现类似的规律。

3结论和启示

不发达国家劳动生产率的提高会对两国贸易产生动态影响。不发达国家劳动人口的增加也会影响两国的贸易模式。有研究表明:当今发达国家的人口增长率约为0.63$\%$,不发达国家的人口增长率约为1.3$\%$。如果发达国家和发展中国家之间的收入差距变小,不发达国家的福利水平会因贸易而进一步上升。在研究动态产品生命周期理论时,还可以假设消费者和供给者是不同质的个体。另外还可以利用现实经济中的数据去检验动态产品生命周期理论。特别是中国作为生产制造大国,中国科学技术的进步,大大缩短了我国学习西方先进技术的周期,甚至体现出后发优势。另外这种研究问题的思路可以应用到我国不同省份的产业消费结构中。我国不同省份存在明显的收入差距,一些高精尖产品的消费在不同省份的普及时间是不一样的。另外,我国不同省份的出口商品特点和结构也不同。这些都可以用动态贸易模型来解释。 
 