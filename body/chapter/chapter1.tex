\chapter{绪论}
\section{研究背景及意义}
%全球性的金融危机已经逐渐过去,在渡过最困难的时期之后,世界整体经济已经出现了起伏上升的复苏趋势,
随着全球经融危机的度过,世界经济已呈现上升趋势,中国在此过程中,成功地抓住机遇,发展自身,已经超越日本成为世界第二大经济体。
在此过程中,以互联网为基础的电子商务在经济发展中起了至关重要的作用。

电子商务主要通过完备的电子通信技术基于网络从事的商业往来,目前各界由于对电子商务的参与程度、理解方向不同,导致对电子商务的定义不明确,但可以确认的是,电子商务依托于信息网络技术,核心是物与物的交换,
操作主体可以看作为交易双方,事实上的主要途径是银行电子支付要点和电子结算要点,客户数据信息内容作为具体支撑点,形成新型的贸易模式。

在互联网高速发展的如今,全世界都已经离不开网络,也离不开电子商务,基于互联网络高速便捷的优点,买家和卖家能够在相距很远的地方甚至可以不用互相知晓地进行商业往来。目前,政府、学术界、商界等公认的电子商务模式可以主要归结为为:B2C,C2C,ABC,B2B,O2O等类型,而近年来B2B类型平台发展尤为迅速。

电子商务打破传统的时空限制,使交易简单、透明,能够节省物力、人力,这些优点越来越受到外贸公司的青睐与关注,他们开始借助电子商务,拓展业务,不少传统的外贸公司因此重视市场的转变,开始进入外贸电子商务之路,投入互联网经济。外贸公司通过电子商务拉动业务,一般采用两种经济手段,一是依托第三方的电子商务平台,租赁自己的虚拟柜台;二是搭建属于自己的独立的外贸网站,掌握流量入口。随着电子商务的渗透影响,越来越多的企业开始搭建自己的外贸电子商务网站。

目前典型的B2B平台有:阿里巴巴、中国制造网、环球资源、ECVV等网站,这种模式比较成熟可靠,也是企业在外贸电子商务采用的较多的方式,同时这类模式的特点是初期费用投入较高,启动资金大,访问量较多,采购商集中,盈利也相应的高很多。

电子商务的重要性不言而喻,这是个大而范的概念,它随着互联网而崛起,发展速度之快令人咋舌。电子贸易的规模化出现在虚拟交易平台出现后,依靠计算机网络的发展,打破了地域、时间、市场规模的限制,通过快速的信息传播、交互,电子商务得以飞速发展,并且潜力巨大,打破了固有的传统交易模式,互联网的日新月异加速了电子贸易的发展。

电子商务依托网络,操作交易方便快捷,既安全但也存在风险,优点与缺点共存,目前看来它的有利之处更为突出,这种集计算机技术、网络技术、信息技术为一体的商务模式发展前景可观,势必会成为将来经济增长的引擎,因此对电子商务的研究非常有必要,本文主要从电子商务的利弊、发展方向等方面着手,浅谈电子商务的未来可能。

通过对电子商务的研究,我们可以充分了解电子商务的概念,剖析电子商务在国际贸易中的重要地位以及深远的影响,分析全球大互联网环境经济中电子商务的优势及不足,为中国开展国际贸易提供理论参考。同时可以发现目前已经存在的问题,及时避免或寻求解决方案。另外需要知道理论与实际存在一定的区别,电子商务的虚拟环境特性需要提供一定的监管手段。
总之,电子商务经济时代已经到来,各国都在努力扶持本国贸易,也都在摸着石头过河,研究电子商务与国际贸易存在迫切的实际需求。

