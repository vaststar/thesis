\chapter{国际贸易的风险和规避措施}
全球经济一体化的发展与互联网科技的发展,让国际贸易变得越来越频繁,贸易对经济发展的推动作用是显而易见的。我国在受到2008年全球经济危机的冲击之后,国际贸易的总体水平稳中有升,为我国的出口贸易企业提供了良好的发展机遇,并取得了非常不错的成绩。但我们仍然需要牢记古人“居安思危”的思想,我们在看到成绩的同时也要看到风险存在,并对这些风险采取一定的规避措施,在风险真的来临时,能够把企业的损失降到最低。

一、国际贸易风险的类型

(一)政策风险

政策风险,主要是指随着全球贸易增多而使贸易摩擦不断地加剧。在国际贸易中,反倾销案件持续增加,技术性贸易壁垒也普遍存在。这让全球的贸易关系也变得越来越复杂:一方面,全球经济一体化的发展让全球贸易趋势呈自由化的方向发展;另一方面,在世界贸易组织的框架内,区域经济的合作获得迅速的发展,各种排他性的贸易保护主义开始不断地以各种看似合理的方式出现。关税是一种保护本国市场发展的贸易壁垒被普遍地使用和接受,变得越来越透明,但是技术性的贸易壁垒,却在一些发达国家出现,将其他国家的竞争企业排除到国门之外。

(二)操作风险

操作风险,主要是指对国际通用的贸易惯例和贸易术语了解的不足。贸易术语很多时候是一种惯例,明确了参与国际贸易各方的权利与义务,如果参与贸易的一方对这些术语不熟悉,就会在贸易中出现操作上的不合规矩。并且一旦被国外的不法商人所利用,就会产生意料之外的贸易纠纷,最严重的就是导致惨重的经济损失。所以,合理地对贸易进行了解和操作十分重要。

(三)汇率风险

国际贸易之间的结算并不像国内贸易一样都使用人民币。那么,不同币种之间进行清算时就存在一个本币与外币的折算比率问题。但比率并不是固定不变的,而是随着国际外汇市场出现波动而有所浮动,这就造成了国际贸易企业各方的实际收入与当初预想的会有不同(贸易中的汇率风险),进而导致外贸企业净利润的增加或者减少。采取一定的措施切实地对外汇风险进行防范,是外贸出口企业必须要重视的问题。

二、国际贸易风险管理的原则

国际贸易中的风险规避管理是在一定原则的指导下进行的,笔者将这些原则总结如下:

(一)风险回避原则

每个外贸企业都有自己独特的产品和擅长领域,风险回避原则就是指出口企业不涉及那些自己不擅长的领域,或者是预测到可能会出现风险的领域。虽然在这种原则的指导下可能会使企业错过一些获得高收益的机会,但是只有采取这种措施才有可能规避较大的风险。

(二)风险抑制原则

风险抑制原则,是指企业通过采取有效的措施对风险进行监控,在风险成形之前就将其扼杀。在对汇率风险进行防范时,风险抑制这个原则是非常适用的,因为很多时候汇率的变化之快,不会给企业留有反应的时间来采取措施挽回损失的。

(三)风险转移原则

参加国际贸易的企业可以通过购买商业贸易保险等方式,将可能产生的风险转移给第三方,这就是风险转移原则。这需要企业的管理者对风险有足够的认识和了解,以便采取合适的投保或者其他转移方式。

三、规避国际贸易风险的有效措施

通过前文对国际贸易风险类型的分析和风险管理的原则,笔者提出了以下对国际贸易风险进行规避的措施:

(一)对贸易伙伴的信用情况进行严查

贸易双方的相互信任和了解是国际贸易中非常重要的一环。为了增进这种信任和了解,为了能够找到信用优质的合作伙伴,需要我们的企业在双方进行合作签字之前就对对方的企业进行严格的调查。这是贸易合作准备工作的一部分,调查的途径主要是通过委托专业的咨询企业进行调查咨询,或者借助银行代理系统对贸易伙伴的信用情况进行调查。

(二)国家加大对人民币汇率的调控力度

关于人民币的汇率都是依据国家的国情进行制定的,常常使用的就是固定利率,但是在国际市场的贸易中,结算方式是双方互相协商的结果。所以,为了我国的企业在协商时有更多的资本,在制定人民币汇率时,就要考虑到具体的国情和人们的利益,尽量地保持汇率的稳定。一些强势的经济体国家会利用国际货币体系来转嫁本国的经济危机,从而得到更多的经济利益。这是我国需要坚决抵制的,尤其是要反对他国使用强制性的手段迫使我国利率的提升。

(三)运用技术手段切实转移风险

在国际贸易进行的过程中,我国的外贸企业可以根据风险管理的原则,使用一些技术手段来有效地规避风险。第一,可以通过合理的投保来降低运输风险。因为国际贸易的路途通常非常遥远,且耗时较长,在整个过程中存在着大量的隐患和风险,如海盗、战争、空难等,企业需要通过办理运输保险来促进企业外贸事业的发展。第二,可以通过妥善地利用国家提供的出口信用保险。这是由国家成立的专门的保险机构来向出口企业提供的一种政策性的保险,承担一些商业保险不愿意承担的保险业务,能够有效地规避贸易中的各种风险。第三,通过积极地引入国际保险业务来提供综合性的财务服务,规避外贸过程中收汇风险等。

(四)遵循国际贸易惯例

国际贸易的惯例通,常是指国际性的贸易组织或者商业社会的团体制定出来的一些贸易准备、规矩。这需要参与国际贸易企业的工作者牢牢地掌握,以便企业在签合同、交货物、结货款时能够严格地按照进出口贸易的规则程序来办事。妥善地处理好合同、单据、货物之间的关系,特别要注重信用证与单据之间的关系,以确保货款能够得到及时的清偿。同时,对国际贸易中的贸易术语进行全面的了解,以免在进行国际贸易的过程中因为不了解而遭受损失,面临“哑巴吃黄连,有苦说不出”的无奈局面。

四、结束语

国际贸易是一项对外部环境有较高依赖性的贸易活动,国家之间的各种关系的变化、政策的变动都能引发国际贸易的动荡。尤其是随着国际市场竞争激烈程度的加剧,各国之间为了本国经济的发展和贸易的顺利进行会出台各种办法,这些因素都让国际贸易变得更加的复杂、多变,导致贸易风险、贸易摩擦事件的频繁发生,并造成了一定损失。但凡事皆有两面性,如果我国的外贸企业能够抓住机遇,积极地进行国际市场的开拓,采取有效的措施来规避风险,那么我国的国际贸易一定会发展得越来越好。