\chapter{电子商务对国际贸易的影响}
电子商务依托于强大的互联网技术,几乎改变了国际贸易的整体构成及贸易方式,冲击着传统贸易,重新定义了国际贸易。随着电子商务的愈加成熟,国际贸易的经营主体以及贸易方式都发生了显著变化:

国际贸易的贸易主体发生了重大变化,
以往的贸易往来主体通过中介进行买卖服务或者担保,而如今只需要通过成熟的网络平台,比如阿里巴巴等B2B平台进行直接交流,省去了中间商,同时买卖双方通过交易平台能够快速定位所需客户。这种贸易模式依托于网络平台公司在其专业领域的领先技术,完成单个公司无法实现的市场功能,进行信息整合推送,承担着信息搜集、处理、传递、推荐功能,完成了企业间的弱联系,改变了以往企业间的贸易模式。

国际贸易的市场交易模式发生了重大变革,
传统市场存在的前提条件是空间上必须有一定的地域,聚集各个商家进行贸易往来,而电子商务另辟蹊径,利用互联网的虚拟特性,以全球的信息网为基础,将全球贸易笼络成一个超级市场,促进了世界经济的全球化,改变了国际贸易市场。

\section{电子商务的积极影响}
电子商务大大压低了国际贸易的贸易成本,电子商务依托于国际互联网,交易双方通过网络即可完成交易,节约了时间成本,而物流的出现



现代信息技术大大降低了国际贸易的经营成本。以国际互联网为核心的电子商务应用不在以时空为界限,电子商务的整合效应也简化了业务的流程,网络第三方服务也日益完美。网上订货、网上促销、网上谈判、跨国公司内部网络销售均为电子商务交易开辟了新形式。信息技术与社会化服务系统结合产生了EDI工程,进出口商利用电子表格进行商品的报关、商检、保险、运输投保和结汇等工作。这种做法大大减少了人才、物力和时间消耗,降低了流通成本和交易费用,减少了中间环节,加快了国际贸易的节奏。


  电子商务突破了时空限制,贸易服务的供应者不必跨出国门甚至家门就能为别的国家的客户提供各种服务。另外,公司足不出户就可以同时接到来自不同国家的业务,而不必担心日程安排和国际旅费等方面的问题。例如:在新产品开发、工业设计、咨询人才培训、医疗诊断等领域,这种扩大的服务需求直接导致国际服务贸易的飞快发展,促进国际贸易商品结构的高端化。

  4.
  1.电子技术层面,应采用一系列行之有效的电子技术方法,并保证交易业务的安全。目前,最普遍的方法有:应用数字时间解决电子交易文件发出的时间认证问题;应用数字证书解决网上交易对象身份认定问题;应用非对称密钥密码技术解决交易信息传输过程中的保密问题;应用数字摘要、数字信封、数字签名等方法解决交易信息传达后的完整、正确、未被修改的验证问题。
  
    2.加强法律法规的研究与制定。现行国际贸易法是基于传统的有纸贸易方式而制定的,所以许多法规不适用于电子商务贸易方式,对电子商务的发展会带来许多难以克服的障碍。为了保证电子商务的良好发展,围绕电子商务发展及相关的信息安全、网络管理、知识产权保护、金融结算、等问题,应加快现行法律法规的修改步伐,及时制
  
    定、出台新的贸易法规。
  
    3.降低过高的资费,改革收费制度。制约电子商务在我国发展的一个重要原因就是高收费。例如:网费,目前我国国内互联网的上网费用比美国高出14倍,国际专线价格更加昂贵,DDN专线不但收取月基本费用,同时收取流量费。改革现行收费制度,建立一个完善的符合市场要求的收费体系,下降过高的资费能有效提高我国公民对电子商务的接受水平,是促进我国电子商务发展的重要条件。
  
    4.建立人性化的国际CRM系统。CRM的核心思想是了解客户所想,满足客户所想,从而提高企业经营绩效。在外贸企业中建立CRM,不但能有效避免传统贸易手段中对业务员的过度依赖,而且能更加高效、方便、快速地实现与前端客户的交互,满足客户服务。
  
    5.积极参与国际合作与对话。对于国际性讨论对话活动,我国政府虽有参与,但不够积极。实际上,这些对话活动不但决定不同国家、地域之间的利益分配,直接影响着电子商务下新交易法律法规的制定,而且加强了世界各国的交流,提高了各国电子商务的国际兼容性。对此,我国政府应予以足够重视。
  
    四、结论
  
    电子商务时代瞬息万变,国际贸易展现出的变化是对新的贸易方式和贸易环境的变现。根据专家评测,20140年全球电子商务交易额将达到一万亿美元。中国作为最大的发展中国家,正面临着应用电子商务促进国内经济发展,正是走向世界的机遇也是挑战。想要更加有效地参加国际市场竞争和在经济全球化过程中获得最大的利益收货。我们应当把电子商务在国际贸易中的姿态放在首位并且借鉴发达国家的经验,培育我国公司在国际贸易中应用电子商务的竞争能力,这将使我国国际贸易发展创建一个新的里程碑。
\section{电子商务存在的问题}

(一)电子商务的积极影响

自从在国际贸易运用了电子商务,国际贸易的复杂流程变得非常简便。相比于传统的国际贸易,如果国际贸易非常频繁,就会使得国际贸易间的手续流程非常繁杂,耽误了很多的时间,存在很多的误差之处,贸易效率低。而随着电子商务的使用,国际贸易立刻发生巨变,因为电子商务是一种无纸交易,把传统贸易模式下的复杂琐屑过程变得非常快捷,得到很多国家的重视。无纸交易是指在现代信息技术的帮助下,通过互联网的快速传播,使得双方的交易信息迅速传达至对方,这样一来就可以减少传统贸易间的订单需求等,快速的达成贸易目的。无纸交易的主要核心是电子数据间的交换,简称EDI,EDI是指利用计算机技术和互联网的方便特点简化国际贸易的业务流程。和传统贸易相比,不需要前期的贸易准备、中期的贸易技术管理人员洽谈和最后贸易订单签订和人员跟进等。因此使用电子商务可以大大减少订单的各种准备工作,提高贸易的效率,同时又除去了不必要的成本投入。

(二)电子商务的消极影响

电子商务带来方便的同时,不可避免也造成了一些消极的影响。首先电子商务加剧了发达国家与不发达国家之间的差距,因为电子商务使用的基础取决于高度发达的电子信息技术和互联网技术等先进的科技技术,而电子商务的快速使用又使得国家间的贸易业务变得更为广阔,加速了国际贸易的发展,提高了国际贸易的效率。因此在这种不公平的发展基础上,发达国家比不发达国家更有优势,发达国家与发展中国家的差距将会越来越大,出现马太效应。其次,相比较于传统的贸易模式,电子商务主导的贸易模式极易造成大量的税款流失。因为电子商务的快速发展,造成了大量贸易的无纸交易,虽然可以加快贸易的效率,但不可否认也在一定程度上影响了发展中国家的税收,导致发展中国家的发展滞后。由于互联网虚拟的特点,很容易使得无纸交易变成欺诈交易,造成国际贸易的混乱,影响国家的正常收税活动,同时由于电子商务的无纸交易,导致了国内中介的衰败,影响国家税收。 




电子商务管理方式在国际贸易经营管理方式中的变化逐渐加大,交互方式网络运行机制对于电子商务而言,其变化程度极其深远,在此种发展背景之下,国际贸易中能够所展现出的最优化配置,借助国际贸易世界经济来完成,此时,若是从全球化角度进行分析,市场机制效能才完美彰显。老旧式贸易模式现在被此种贸易形势替代掉,四流一体成为主流,简而言之,就是将物流作为支撑,将资金流视为基本运作形势,商品流通是主体,创新整改后的经营管理模式特性正是如此,此种经营模式在实施商业贸易交流时将信息网络作为依据,并且站在不同角度之上实施真正意义上的互动。我们应该意识到,网络可以使生产者和用户以及消费者之间实现互动,供货制度和零库存生产,二者才能得以完美实现,商品流动性就会变得更加的符合市场规律,与社会需求相互契合,符合经济发展诉求,中间商就是我们所熟知的信息网络。

三、现存问题要点分析

电子商务若想得到长远发展与立足,应该借助信息基础建设来完成目标,但是现有信息基础设施建设还有待增强,并且此时的网络规模局限性十分明显。重要的网络安全问题没有透彻得到解决,网络对于电子商务而言,十分重要,网络数据传递、网络数据交换如此,安全性处理亦如是。商业交易在网络上进行,安全性未能得到透彻保障,当前我国经济出在转型时期,尚未形成较为成熟的市场,社会信用结构体系没有得到透彻完善,网络交易受影响因素主要涵盖了身份识别内容、商业机密内容和通讯安全内容,尤其是没有授权的信息非法篡改和信息内容恶意拦截,网络交易和客户信息记录保存等仍有些许漏洞。电子商务对老旧式商业法律和商业习惯造成冲击,老旧式商事立法和商事习惯中,有纸式模式成为主流,但是电子商务过度的去依赖网络空间,理解为不完全在纸张基础上进行交易,因此若是站在客观角度进行研究,老旧式立法办法和电子商务运作间从根本上讲没有联系。但是电子商务这一事项上所涉法律事务繁多,风险仍在,那么电子商务在具体应用阶段就会滋生诸多障碍因素。虽然当前网络信息和电子商务在我国有了法律条文规定,但权威性法律甚是匮乏,争端出现其实没有透彻解决的可能,这样一来,电子商务发展稳定性和电子商务发展快速性,二者都会受到影响。

四、解决办法分析

(一)顺势而为

全球经济一体化进程日渐加快的今天,国际贸易领域范围内,信息技术应用频率有所加快,世界上每个国家在发展商务的整个阶段内,电子商务都发挥着巨大效能,我国若想在本世纪竞争中独占鳌头,就必须将电子商务作为核心研究对象,只有这样,才能在一定程度上与发达国家“抗衡”且具备一同进步和发展的机会。

(二)依法而行

电脑空间法律实际上是世界性的,本世纪中国国际贸易是一个发展快速的阶段与过程,自身具备发展前景,电子商务此时就必须做出改变,国际贸易竞争也要向发达国家看齐,与国际立法去向靠拢,并立足于国际立法标准。电子商务发展,务必要与法律之间达成匹配,按照电子商务发展基本需求,立法部门应该强化电子提单管理力度和司法官权的强化力度,对合同法和票据法以及消费者权益保护法进行健全,在安全认证方面务必加大规范力度,在线支付与电子商务机构和企业管理方面也要重视起来,不同类型的电子商务贸易都要透彻规范和整改。

(三)由浅入深

基础建设要做到位,为电子商务发展提供强劲的物质基础支持,国家方面电子商务现在已经得到了大量影响,信息基础设施建设优良决定着机会和空间的发展效果,所以为了促进电子商务更好更优的发展,我们应该注重信息基础设施建设,与此同时,提升网络资源利用效率,长此以往,行业分割管理制度会愈加完善,体制日渐健全,资源效益会得到质的提升。中国经济发展,电子商务方面会受到过高收费限制。

五、结束语

今后国家贸易发展,电子商务在其中起到了至关重要的效能,外贸企业务必要坚持自身原则,提升竞争力才是根本,要对电子商务进行深度认知,建立站点,供求信息扩大,建立客户群和交流渠道,树立优质企业形象,在国家贸易中利用电子商务手段谋求更为广阔的天地。 



 
 
 

3 现代物流对我国对外贸易的促进作用

3.1 降低了经营成本

现代物流行业的发展,对我国对外贸易活动的开展具有极强的促进作用。最直接地体现在降低了经营的成本。一般而言,要想提升贸易效益,首先应该综合考虑各国在经济方面所具有的优势。而优势则体现在对成本的控制与比较方面。但针对目前国际贸易发展现状分析,经济的高速发展使得成本降低的空间越来越小,可是在物流方面可以压缩的空间却越来越大。相关数据表明,我国物流行业及其相关行业一年的总支出大约为19000亿元,而物流经济效益在我国GDP总量中所占比重高达25%。故而,通过物流活动中各个环节的协调,可以使物流行业的经营成本不断降低,从而降低国际贸易成本,提升国际贸易效益与水平。

3.2 扩大了贸易规模

现代物流的发展也使得国际贸易的规模不断扩大,因为目前我国的对外贸易伙伴处于相对集中的状态,且大多为欧亚以及美洲国家。这足以看出我国对国际贸易的依赖程度相当高,不利于将市场扩展。而现代物流的发展则对国际贸易的规模有重要影响,尤其是现代物流发展中,口岸物流的建设以及交通运输网络的完善、电子信息技术的进步等,都为我国国际贸易活动的开展以及货品的运输提供了条件,从而扩大了我国国际贸易的整体规模。

3.3 加快市场反应速度的同时也提升了整个市场竞争力

我国市场中的商品普遍存在技术含量较低、劳动力投入较大的特点。也正是因为我国这种传统的产品生产形式,我国的产品对外竞争力较弱。为了有效改善这种现象,不仅需要增加产品的研发力度,另外还需要通过科学有效的管理,加快进出口企业对市场的反应速度。既需要降低交易成本,还需要生产适销对路、技术含量较高的产品,实现我国进出口经济的协调发展,从而提供较好的产销服务,提升我国产品在国际贸易竞争中的竞争实力。

探究现代物流对国际贸易的促进作用,实际上也是实现我国现代物流发展、促进我国国际贸易进步的重要举措。要想转变为贸易大国,现代物流企业在其中的重要性不言而喻。不仅需要从多方面进行改变,同时还需要历经长时间的付出。作为现代物流企业,要明确与国际贸易之间彼此的促进作用,这样将有利于实现现代物流对国际贸易的推动作用。

4 我国现代物流行业发展的对策

4.1 转变观念与意识,提高服务质量

作为现代物流企业,必须对传统的物流服务方式与手段进行创新。物流企业因为规模较小,所以往往缺乏相应的竞争实力。故而需要我国物流企业的领导人员转变自身的意识与观念,从而不断提升自身企业的服务质量。既需要为货主提供高效率的物流服务,还需要具有个性化服务特点,满足客户个性化的消费需要与服务需要。

4.2 加快现代物流基础设施建设

目前我国物流基础设施建设还相对落后,与国际贸易中需要实现的物流机械化、装备现代化要求较远。所以要想更好地发展我国现代物流行业,并且使之与国际贸易发展相契合,加大物流基础设施建设刻不容缓。比如可以建设集装箱专用码头,这是现代物流行业发展的基础。同时更应该满足国际物流连续性作业的要求,引进众多国外先进技术,实现物流设备的引进、研制、发展,从而提高货物的运输效率与水平。

4.3 培养专业物流人才

培养专业物流人才,是我国现代物流行业在未来的发展过程当中重要的基础和保障。所以许多院校也可以开设物流相关课程,其目的就是为物流行业提供更多既具有较强理论性,又具有极强操作性的专业人才。不仅需要大力引进国内外优秀人才,还需要建立起相应的培训与奖励制度,通过众多的方式留住优秀人才,进而实现我国现代物流企业的长久进步,为我国国际贸易的发展带来基础性帮助。

4.4 政府和国家给予足够的扶持与支持

政府和国家的支持,是现代物流企业发展的重要保障。尤其我国现代物流行业发展还处于初步发展阶段,所以往往需要国家的支持,这样才能规范我国的贸易市场,实行严格的进出口原则,从而让现代物流企业为国际贸易发展服务。

5 结 论

现代物流行业在未来的发展中,必将呈现出蓬勃发展的态势,同时其自身对国际贸易的影响也将越来越大。因此,现代物流企业必须通过合理的方式对自身进行创新与完善,制定出与国际贸易发展和市场经济相适应的对策,从而为世界贸易的繁荣与进步提供基础与条件。













全球经济一体化的发展与互联网科技的发展,让国际贸易变得越来越频繁,贸易对经济发展的推动作用是显而易见的。我国在受到2008年全球经济危机的冲击之后,国际贸易的总体水平稳中有升,为我国的出口贸易企业提供了良好的发展机遇,并取得了非常不错的成绩。但我们仍然需要牢记古人“居安思危”的思想,我们在看到成绩的同时也要看到风险存在,并对这些风险采取一定的规避措施,在风险真的来临时,能够把企业的损失降到最低。

一、国际贸易风险的类型

(一)政策风险

政策风险,主要是指随着全球贸易增多而使贸易摩擦不断地加剧。在国际贸易中,反倾销案件持续增加,技术性贸易壁垒也普遍存在。这让全球的贸易关系也变得越来越复杂:一方面,全球经济一体化的发展让全球贸易趋势呈自由化的方向发展;另一方面,在世界贸易组织的框架内,区域经济的合作获得迅速的发展,各种排他性的贸易保护主义开始不断地以各种看似合理的方式出现。关税是一种保护本国市场发展的贸易壁垒被普遍地使用和接受,变得越来越透明,但是技术性的贸易壁垒,却在一些发达国家出现,将其他国家的竞争企业排除到国门之外。

(二)操作风险

操作风险,主要是指对国际通用的贸易惯例和贸易术语了解的不足。贸易术语很多时候是一种惯例,明确了参与国际贸易各方的权利与义务,如果参与贸易的一方对这些术语不熟悉,就会在贸易中出现操作上的不合规矩。并且一旦被国外的不法商人所利用,就会产生意料之外的贸易纠纷,最严重的就是导致惨重的经济损失。所以,合理地对贸易进行了解和操作十分重要。

(三)汇率风险

国际贸易之间的结算并不像国内贸易一样都使用人民币。那么,不同币种之间进行清算时就存在一个本币与外币的折算比率问题。但比率并不是固定不变的,而是随着国际外汇市场出现波动而有所浮动,这就造成了国际贸易企业各方的实际收入与当初预想的会有不同(贸易中的汇率风险),进而导致外贸企业净利润的增加或者减少。采取一定的措施切实地对外汇风险进行防范,是外贸出口企业必须要重视的问题。

二、国际贸易风险管理的原则

国际贸易中的风险规避管理是在一定原则的指导下进行的,笔者将这些原则总结如下:

(一)风险回避原则

每个外贸企业都有自己独特的产品和擅长领域,风险回避原则就是指出口企业不涉及那些自己不擅长的领域,或者是预测到可能会出现风险的领域。虽然在这种原则的指导下可能会使企业错过一些获得高收益的机会,但是只有采取这种措施才有可能规避较大的风险。

(二)风险抑制原则

风险抑制原则,是指企业通过采取有效的措施对风险进行监控,在风险成形之前就将其扼杀。在对汇率风险进行防范时,风险抑制这个原则是非常适用的,因为很多时候汇率的变化之快,不会给企业留有反应的时间来采取措施挽回损失的。

(三)风险转移原则

参加国际贸易的企业可以通过购买商业贸易保险等方式,将可能产生的风险转移给第三方,这就是风险转移原则。这需要企业的管理者对风险有足够的认识和了解,以便采取合适的投保或者其他转移方式。

三、规避国际贸易风险的有效措施

通过前文对国际贸易风险类型的分析和风险管理的原则,笔者提出了以下对国际贸易风险进行规避的措施:

(一)对贸易伙伴的信用情况进行严查

贸易双方的相互信任和了解是国际贸易中非常重要的一环。为了增进这种信任和了解,为了能够找到信用优质的合作伙伴,需要我们的企业在双方进行合作签字之前就对对方的企业进行严格的调查。这是贸易合作准备工作的一部分,调查的途径主要是通过委托专业的咨询企业进行调查咨询,或者借助银行代理系统对贸易伙伴的信用情况进行调查。

(二)国家加大对人民币汇率的调控力度

关于人民币的汇率都是依据国家的国情进行制定的,常常使用的就是固定利率,但是在国际市场的贸易中,结算方式是双方互相协商的结果。所以,为了我国的企业在协商时有更多的资本,在制定人民币汇率时,就要考虑到具体的国情和人们的利益,尽量地保持汇率的稳定。一些强势的经济体国家会利用国际货币体系来转嫁本国的经济危机,从而得到更多的经济利益。这是我国需要坚决抵制的,尤其是要反对他国使用强制性的手段迫使我国利率的提升。

(三)运用技术手段切实转移风险

在国际贸易进行的过程中,我国的外贸企业可以根据风险管理的原则,使用一些技术手段来有效地规避风险。第一,可以通过合理的投保来降低运输风险。因为国际贸易的路途通常非常遥远,且耗时较长,在整个过程中存在着大量的隐患和风险,如海盗、战争、空难等,企业需要通过办理运输保险来促进企业外贸事业的发展。第二,可以通过妥善地利用国家提供的出口信用保险。这是由国家成立的专门的保险机构来向出口企业提供的一种政策性的保险,承担一些商业保险不愿意承担的保险业务,能够有效地规避贸易中的各种风险。第三,通过积极地引入国际保险业务来提供综合性的财务服务,规避外贸过程中收汇风险等。

(四)遵循国际贸易惯例

国际贸易的惯例通,常是指国际性的贸易组织或者商业社会的团体制定出来的一些贸易准备、规矩。这需要参与国际贸易企业的工作者牢牢地掌握,以便企业在签合同、交货物、结货款时能够严格地按照进出口贸易的规则程序来办事。妥善地处理好合同、单据、货物之间的关系,特别要注重信用证与单据之间的关系,以确保货款能够得到及时的清偿。同时,对国际贸易中的贸易术语进行全面的了解,以免在进行国际贸易的过程中因为不了解而遭受损失,面临“哑巴吃黄连,有苦说不出”的无奈局面。

四、结束语

国际贸易是一项对外部环境有较高依赖性的贸易活动,国家之间的各种关系的变化、政策的变动都能引发国际贸易的动荡。尤其是随着国际市场竞争激烈程度的加剧,各国之间为了本国经济的发展和贸易的顺利进行会出台各种办法,这些因素都让国际贸易变得更加的复杂、多变,导致贸易风险、贸易摩擦事件的频繁发生,并造成了一定损失。但凡事皆有两面性,如果我国的外贸企业能够抓住机遇,积极地进行国际市场的开拓,采取有效的措施来规避风险,那么我国的国际贸易一定会发展得越来越好。