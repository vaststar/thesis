\chapter{电子商务对国际贸易的影响与未来发展}
电子商务依托于强大的互联网技术,几乎改变了国际贸易的整体构成及贸易方式,冲击着传统贸易,重新定义了国际贸易。随着电子商务的愈加成熟,国际贸易的经营主体以及贸易方式都发生了显著变化:

国际贸易的贸易主体发生了重大变化,
以往的贸易往来主体通过中介进行买卖服务或者担保,而如今只需要通过成熟的网络平台,比如阿里巴巴等B2B平台进行直接交流,省去了中间商,同时买卖双方通过交易平台能够快速定位所需客户。这种贸易模式依托于网络平台公司在其专业领域的领先技术,完成单个公司无法实现的市场功能,进行信息整合推送,承担着信息搜集、处理、传递、推荐功能,完成了企业间的弱联系,改变了以往企业间的贸易模式\cite{zhuyahui2013}。

国际贸易的市场交易模式发生了重大变革,
传统市场存在的前提条件是空间上必须有一定的地域,聚集各个商家进行贸易往来,而电子商务另辟蹊径,利用互联网的虚拟特性,以全球的信息网为基础,将全球贸易笼络成一个超级市场,促进了世界经济的全球化,改变了国际贸易市场。

\section{电子商务的积极影响}
2016年跨境电商整体的交易规模达到6.3万亿,而这个数字预计将在今年达到8.8万亿之多,由此可见电子商务对经济增长的促进之大,因此电子商务能够大大提高贸易量,增加贸易往来。

同时,电子商务大大压低了国际贸易的贸易成本,依托于国际互联网,交易双方通过网络即可完成交易,节约了时间成本,而现代国际物流的发展加速了贸易完成的速度,降低了电子商务的时间成本,提升了电子商务的效率。

自从在国际贸易中引入了电子商务模式,以往国际贸易的复杂流程现在变得非常简便,
以贸易发生频率举例,如果贸易频繁,由于传统国际贸易中的跨境流程手续复杂,时间消耗成本巨大,同时人为参与因素过多,误差多,导致贸易效率低下,
但是随着电子商务模式的介入,引入电子化手段,采用无纸交易,去除了繁琐的中间过程,立刻改变了这种低效贸易模式,大大加速了经济流通速度。

同时,传统的国际贸易中大量数据的存储很不方便,确认也很慢,电子商务使用互联网交易,数据通过网络发送,交易双方的信息存取速度快,交易确认及时。
另外网络信息时代的弱智能化手段,使交易有据可寻,资料准备可以采用电子模板形式,交易的达成只需要针对特定的需求填充模板即可,这样面对不同的客户,卖方只需要修改少量信息,大大减少了订单、备货等的前期准备工作,节省了不必要的成本投入。

\section{电子商务的消极影响}

电子商务在带来方便、促进经济的同时,不可避免地产生了一些消极的作用。
由于电子商务立足于电子信息、互联网科技等先进技术,而发达国家与不发达国家地互联网技术有着显著的差距,这种技术上地不对等导致了贸易顺差,
使技术越先进、互联网越普及的国家越能充分利用电子商务发展自身经济,这样循环往复形成马太效应,互联网强国愈发领先那些技术不发达的国家。

电子商务模式的无纸化交易特性可能会引起税款流失,由于互联网经济的迅速发展,交易发生迅速频繁,同时相应的法律等监管措施没有相应地完善,电子商务交易的应收税额无法征取,另外,电信类诈骗等违法行为借由互联网得以滋生,电子贸易中存在大量诈骗、欺诈行为,不法分子可以利用特殊的电子技术伪造信息进行欺诈,影响社会安定团结。

电子商务对传统销售行业带来了一定的冲击,足不出户就能购买到任何产品,更加符合现代年轻人的选择,人们可以利用碎片化的时间上网解决购物需求,比如电子产品、衣服等生活用品,而不必专门抽出大量时间到实体店购买。这导致线下的实体门店大量关闭,存活下来的更多是大型商场。
\section{电子商务的未来发展}
电子商务既促进了经济增长,提高了贸易效率,也存在着不可忽视的问题,因此需要针对已经存在以及可能存在的问题提出相应的解决方案。电子商务的根基是网络环境,而我国的互联网科技实力目前相对落后于以美国为代表的发达国家,信息基础设施有待增强,并且网络中数据传递的安全性需要尤其提高,可以以数学为基础,寻找更安全的加密方式,也可以利用u盾等物理设备与网络相结合,提升安全系数,甚至网络上的交易双方强制实名制,借助公安系统打击不法交易。

互联网本质上是全球的,那么相应的互联网相关的法律也应该全球化,至少我国的互联网法律可以向发达国家看齐,向国际立法靠拢,必须提供相应的法律安全条款,加大监管制约力度,才能拥有安全、积极向上、健康的电子商务。

电子商务的最终环节是物流环节,那么通过加强物流建设与监管,既可以提高电子商务的运行效率,又可以跳过互联网的虚拟特性,在实体现实中实现对电子贸易的监管调控。

另外,中国目前针对线下实体经济受到冲击的问题,提出了互联网+的概念,将实体经济与互联网结合,形成新的电子商务模式,用户可以在互联网上购买某品牌,然后去线下对应的实体店进行取货;外卖行业的兴起也正是电子商务的功劳,用户网上点餐后,商家准备好商品,骑手到商家取货送到用户手中,完成交易闭环,利用互联网盘活线下实体经济。

我们需要发挥创新精神,大力培养优秀人才,目前中国的移动支付领域迅速崛起,遥遥领先世界,正是因为中国的社会主义环境不同于资本主义国家,国家信用体系尚未完善,信用卡还未普及,国内的互联网巨头看到了另一种电子商务的机会,并迅速变现。因此我们要继续发扬创新精神,取得更大更好的成就。
%\section{本章小结}
%本章探讨了电子商务对国际贸易的积极影响及存在的问题以及未来的发展方向,电子商务在国际贸易中的重要作用毋庸置疑,有利有弊。我们要避免弊端,利益最大化,发扬创新精神,加快互联网发展进程,完善相关法律法规,加强物流建设,培养专业人才,才能够在电子商务领域后来者居上。