\chapter{电子商务的影响与发展}
电子商务依托于强大的互联网技术,几乎改变了国际贸易的整体构成及贸易方式,冲击着传统贸易,重新定义了国际贸易。随着电子商务的愈加成熟,国际贸易的经营主体以及贸易方式都发生了显著变化:

国际贸易的贸易主体发生了重大变化,
以往的贸易往来主体通过中介进行买卖服务或者担保,而如今只需要通过成熟的网络平台,比如阿里巴巴等B2B平台进行直接交流,省去了中间商,同时买卖双方通过交易平台能够快速定位所需客户。这种贸易模式依托于网络平台公司在其专业领域的领先技术,完成单个公司无法实现的市场功能,进行信息整合推送,承担着信息搜集、处理、传递、推荐功能,完成了企业间的弱联系,改变了以往企业间的贸易模式。

国际贸易的市场交易模式发生了重大变革,
传统市场存在的前提条件是空间上必须有一定的地域,聚集各个商家进行贸易往来,而电子商务另辟蹊径,利用互联网的虚拟特性,以全球的信息网为基础,将全球贸易笼络成一个超级市场,促进了世界经济的全球化,改变了国际贸易市场。

\section{电子商务的积极影响}
2016年跨境电商整体的交易规模达到6.3万亿,而这个数字预计将在今年达到8.8万亿之多,由此可见电子商务对经济增长的促进之大,因此电子商务能够大大提高贸易量,增加贸易往来。

同时,电子商务大大压低了国际贸易的贸易成本,依托于国际互联网,交易双方通过网络即可完成交易,节约了时间成本,而现代国际物流的发展加速了贸易完成的速度,降低了电子商务的时间成本,提升了电子商务的效率。

自从在国际贸易中引入了电子商务模式,以往国际贸易的复杂流程现在变得非常简便,
以贸易发生频率举例,如果贸易频繁,由于传统国际贸易中的跨境流程手续复杂,时间消耗成本巨大,同时人为参与因素过多,误差多,导致贸易效率低下,
但是随着电子商务模式的介入,引入电子化手段,采用无纸交易,去除了繁琐的中间过程,立刻改变了这种低效贸易模式,大大加速了经济流通速度。

同时,传统的国际贸易中大量数据的存储很不方便,确认也很慢,电子商务使用互联网交易,数据通过网络发送,交易双方的信息存取速度快,交易确认及时。
另外网络信息时代的弱智能化手段,使交易有据可寻,资料准备可以采用电子模板形式,交易的达成只需要针对特定的需求填充模板即可,这样面对不同的客户,卖方只需要修改少量信息,大大减少了订单、备货等的前期准备工作,节省了不必要的成本投入。

\section{电子商务的消极影响}

电子商务在带来方便、促进经济的同时,不可避免地产生了一些消极的作用。
首先,由于电子商务立足于电子信息、互联网科技等先进技术,而发达国家与不发达国家地互联网技术有着显著的差距,这种技术上地不对等导致了贸易顺差,
使技术越先进、互联网越普及的国家越能充分利用电子商务发展自身经济,这样循环往复形成马太效应,互联网强国愈发领先那些技术不发达的国家。

其次,电子商务贸易模式可能会引起税款流失,由于互联网经济的迅速发展,其虚拟特性导致无纸化交易,同时相应的法律监管等没有跟上,交易量

相比较于传统的贸易模式,电子商务主导的贸易模式极易造成大量的税款流失。因为电子商务的快速发展,造成了大量贸易的无纸交易,虽然可以加快贸易的效率,但不可否认也在一定程度上影响了发展中国家的税收,导致发展中国家的发展滞后。由于互联网虚拟的特点,很容易使得无纸交易变成欺诈交易,造成国际贸易的混乱,影响国家的正常收税活动,同时由于电子商务的无纸交易,导致了国内中介的衰败,影响国家税收。 




电子商务管理方式在国际贸易经营管理方式中的变化逐渐加大,交互方式网络运行机制对于电子商务而言,其变化程度极其深远,在此种发展背景之下,国际贸易中能够所展现出的最优化配置,借助国际贸易世界经济来完成,此时,若是从全球化角度进行分析,市场机制效能才完美彰显。老旧式贸易模式现在被此种贸易形势替代掉,四流一体成为主流,简而言之,就是将物流作为支撑,将资金流视为基本运作形势,商品流通是主体,创新整改后的经营管理模式特性正是如此,此种经营模式在实施商业贸易交流时将信息网络作为依据,并且站在不同角度之上实施真正意义上的互动。我们应该意识到,网络可以使生产者和用户以及消费者之间实现互动,供货制度和零库存生产,二者才能得以完美实现,商品流动性就会变得更加的符合市场规律,与社会需求相互契合,符合经济发展诉求,中间商就是我们所熟知的信息网络。

三、现存问题要点分析

电子商务若想得到长远发展与立足,应该借助信息基础建设来完成目标,但是现有信息基础设施建设还有待增强,并且此时的网络规模局限性十分明显。重要的网络安全问题没有透彻得到解决,网络对于电子商务而言,十分重要,网络数据传递、网络数据交换如此,安全性处理亦如是。商业交易在网络上进行,安全性未能得到透彻保障,当前我国经济出在转型时期,尚未形成较为成熟的市场,社会信用结构体系没有得到透彻完善,网络交易受影响因素主要涵盖了身份识别内容、商业机密内容和通讯安全内容,尤其是没有授权的信息非法篡改和信息内容恶意拦截,网络交易和客户信息记录保存等仍有些许漏洞。电子商务对老旧式商业法律和商业习惯造成冲击,老旧式商事立法和商事习惯中,有纸式模式成为主流,但是电子商务过度的去依赖网络空间,理解为不完全在纸张基础上进行交易,因此若是站在客观角度进行研究,老旧式立法办法和电子商务运作间从根本上讲没有联系。但是电子商务这一事项上所涉法律事务繁多,风险仍在,那么电子商务在具体应用阶段就会滋生诸多障碍因素。虽然当前网络信息和电子商务在我国有了法律条文规定,但权威性法律甚是匮乏,争端出现其实没有透彻解决的可能,这样一来,电子商务发展稳定性和电子商务发展快速性,二者都会受到影响。

四、解决办法分析

(一)顺势而为

全球经济一体化进程日渐加快的今天,国际贸易领域范围内,信息技术应用频率有所加快,世界上每个国家在发展商务的整个阶段内,电子商务都发挥着巨大效能,我国若想在本世纪竞争中独占鳌头,就必须将电子商务作为核心研究对象,只有这样,才能在一定程度上与发达国家“抗衡”且具备一同进步和发展的机会。

(二)依法而行

电脑空间法律实际上是世界性的,本世纪中国国际贸易是一个发展快速的阶段与过程,自身具备发展前景,电子商务此时就必须做出改变,国际贸易竞争也要向发达国家看齐,与国际立法去向靠拢,并立足于国际立法标准。电子商务发展,务必要与法律之间达成匹配,按照电子商务发展基本需求,立法部门应该强化电子提单管理力度和司法官权的强化力度,对合同法和票据法以及消费者权益保护法进行健全,在安全认证方面务必加大规范力度,在线支付与电子商务机构和企业管理方面也要重视起来,不同类型的电子商务贸易都要透彻规范和整改。

(三)由浅入深

基础建设要做到位,为电子商务发展提供强劲的物质基础支持,国家方面电子商务现在已经得到了大量影响,信息基础设施建设优良决定着机会和空间的发展效果,所以为了促进电子商务更好更优的发展,我们应该注重信息基础设施建设,与此同时,提升网络资源利用效率,长此以往,行业分割管理制度会愈加完善,体制日渐健全,资源效益会得到质的提升。中国经济发展,电子商务方面会受到过高收费限制。

五、结束语

今后国家贸易发展,电子商务在其中起到了至关重要的效能,外贸企业务必要坚持自身原则,提升竞争力才是根本,要对电子商务进行深度认知,建立站点,供求信息扩大,建立客户群和交流渠道,树立优质企业形象,在国家贸易中利用电子商务手段谋求更为广阔的天地。 


4 我国现代物流行业发展的对策

4.1 转变观念与意识,提高服务质量

作为现代物流企业,必须对传统的物流服务方式与手段进行创新。物流企业因为规模较小,所以往往缺乏相应的竞争实力。故而需要我国物流企业的领导人员转变自身的意识与观念,从而不断提升自身企业的服务质量。既需要为货主提供高效率的物流服务,还需要具有个性化服务特点,满足客户个性化的消费需要与服务需要。

4.2 加快现代物流基础设施建设

目前我国物流基础设施建设还相对落后,与国际贸易中需要实现的物流机械化、装备现代化要求较远。所以要想更好地发展我国现代物流行业,并且使之与国际贸易发展相契合,加大物流基础设施建设刻不容缓。比如可以建设集装箱专用码头,这是现代物流行业发展的基础。同时更应该满足国际物流连续性作业的要求,引进众多国外先进技术,实现物流设备的引进、研制、发展,从而提高货物的运输效率与水平。

4.3 培养专业物流人才

培养专业物流人才,是我国现代物流行业在未来的发展过程当中重要的基础和保障。所以许多院校也可以开设物流相关课程,其目的就是为物流行业提供更多既具有较强理论性,又具有极强操作性的专业人才。不仅需要大力引进国内外优秀人才,还需要建立起相应的培训与奖励制度,通过众多的方式留住优秀人才,进而实现我国现代物流企业的长久进步,为我国国际贸易的发展带来基础性帮助。

4.4 政府和国家给予足够的扶持与支持

政府和国家的支持,是现代物流企业发展的重要保障。尤其我国现代物流行业发展还处于初步发展阶段,所以往往需要国家的支持,这样才能规范我国的贸易市场,实行严格的进出口原则,从而让现代物流企业为国际贸易发展服务。

5 结 论

现代物流行业在未来的发展中,必将呈现出蓬勃发展的态势,同时其自身对国际贸易的影响也将越来越大。因此,现代物流企业必须通过合理的方式对自身进行创新与完善,制定出与国际贸易发展和市场经济相适应的对策,从而为世界贸易的繁荣与进步提供基础与条件。


四、结束语

国际贸易是一项对外部环境有较高依赖性的贸易活动,国家之间的各种关系的变化、政策的变动都能引发国际贸易的动荡。尤其是随着国际市场竞争激烈程度的加剧,各国之间为了本国经济的发展和贸易的顺利进行会出台各种办法,这些因素都让国际贸易变得更加的复杂、多变,导致贸易风险、贸易摩擦事件的频繁发生,并造成了一定损失。但凡事皆有两面性,如果我国的外贸企业能够抓住机遇,积极地进行国际市场的开拓,采取有效的措施来规避风险,那么我国的国际贸易一定会发展得越来越好。