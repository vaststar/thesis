\chapter{浅议电子商务对国际贸易的影响及对策}
一、电子商务的简介

电子商务是利用发达的通信技术通过电子网络进行的商业活动。政府、学术界、企业界因参与电子商务的程度不同,对其定义的说法不一。电子商务依托信息网络技术,核心是物物交换;也被解释为互联网、企业网和增值网络的电子交易活动及其相关的业务,是传统商业在各个方面的电子网络信息技术化。在互联网高速扩展的今天,全世界已经离不开网络,基于互联网络高速便捷的特点,买家和卖家远距离地进行各种商务活动,从而满足消费者的购物需求,是店铺的网上交易和电子支付的一种新形式的商业交易活动,其中包括相关的金融服务。政府、学术界、商界公认的电子商务模式分为:ABC,B2B,B2C,C2C,B2M、M2C、B2G、C2G,O2O等。

1.电子商务具备的特点。

(1)电子商务作为一种新的经营方式具有通用性,生产企业和分销企业、最终消费者、政府都是数字化世界的一员。(2)在电子商务环境下,人们不再受到时间地点的限制,在过去看来复杂的业务活动,客户可以以非常简单的方式在短时间内完成,可以随时随地通过互联网检测银行账户资金信息等,并使企业对客户服务的质量较以前有很大的提高。在电子商务业务中,有大量的网络资源可供开发和通信,工作人员的工作时间地点灵活,具有便捷性。(3)电子商务可以规范工作事务的处理流程,将电子信息的处理和工作人员操作组成一个不可分割的整体,这样不仅可以提高资源利用率,而且可以提高系统的严谨性。(4)在电子商务中,安全是一个时时刻刻都要引起重视的核心问题,需要网络可以提一套完整且安全系数高的解决方案,如加密签名、访问控制、防病毒等,这是和传统的业务非常不同的。(5)电子商务活动本身是一个协调的过程,内部客户、公司、生产商、批发商、零售商都是机器运行不可或缺的零件。电子商务正常通畅地运行还需要银行、物流部门、通信中心、技术服务人员等相互默契合作,协调完成。

2.电子商务的功能。

电子商务可以提供的服务包括互联网交易和其他方面全过程的管理服务。因此,它具有宣传广告、洽谈询问、网上订购和支付、服务传递、咨询询问、交易管理等功能。

二、电子商务对国际贸易的具体影响与表现

电子商务的出现,对国际贸易的影响不断深入,在国际贸易、运行机制、竞争态势等方面都带来不可低估的影响。具体表现如下:

1.电子商务影响国际贸易方式。

全球市场以企业的信息网络为纽带形成统一的整体,促进世界市场经济全球化的形成。大数据应用于电子商务中,国际贸易的物流系统由单一向立体化转变,这样物流体系中每一步都会得到高效率的管理。传统的贸易物流模式已不能很好地满足当前电子商务发展的需要,国际贸易运输方式需要不断开展新型创新的运输方式进行运作。

2.电子商务影响着国际贸易的物流模式。

电子商务对国际贸易传统的物流模式产生很大的影响,国际贸易的电子商务改变传统国际贸易以单向物流为主的运作方式,贸易活动利用互联网完成,即在网上完成信息、商家、资金链和物流的完美协调。直接贸易成为重要贸易方式。电子商务扩展整个国际贸易市场,节省交易者所需的贸易时间,缩小贸易者之间交易的空间,使得服务变得更周到、全面。不仅如此,来自不同国家的消费者在同一时间向同一家国际贸易主体进行业务上的咨询与洽谈,更推动国际贸易的发展步伐。

3.电子化是未来发展趋势。

根据目前电子商务对国际贸易的影响程度,经过分析可以预测“无纸贸易”将成为国际贸易的发展趋势。国际贸易电子化和电子商务相结合的产物正处于萌芽状态,所以没有统一的定义,EDI通常也被称为无纸贸易,即电子化国际贸易,可以理解为交易当事人或者那些参与者利用网络信息技术和现代计算机网络在全球进行各种业务活动,包括货物交换、服务贸易和技术等,电子化的国际贸易将会是一个流畅协调的业务流程。

4.电子商务影响国际贸易的运行机制。

电子商务影响传统的流通模式,有利于缩短中间环节,大大降低生产者和消费群体之间沟通的难度,改变传统意义上的市场架构,大大方便生产者和消费群体之间直接交易,显着降低管理成本并逐渐提高国际经济与贸易的效率和效益。互联网环境下的电子商务运作机制,为国际贸易的发展提供了信息更完整的市场环境,使得信息和资源共享跨境转移得以实现,可以在一定程度上满足国际贸易不断增长的需求,推动国际经济与贸易的向前发展。有了电子商务参与的国际贸易成本降低,使得成本太高或难以执行的电子商务交易成为可能。电子商务的发展,促使虚拟公司的出现。在各自专业领域的现代虚拟企业拥有信息和通信技术、核心技术等,通过公司集团的网络运营互连“虚拟经营”完成企业无法承担的市场功能,从而使得市场提供更有效的服务和供货操作。这种虚拟经营的经济环境,迎合了厂家竞争和消费者的个性需求向多元化方向发展,实现了优势互补、资源互用、好处互相分享的局面。在电子商务环境下,运用现代信息技术,单一的外贸公司联合形成合作组织――虚拟电子社会,实现单个外贸公司无法实现的市场的功能。

三、针对措施

传统的国际贸易模式被替代已经成为不可逆转的趋势。面临当前严峻的挑战,企业必须加快转型步伐,主动参与国际竞争,采取措施提升自身竞争力。为了更好地展现我国企业的实力,使我国离成为对外贸易强国的目标更进一步,要采取以下措施。   1.加快转型步伐,迎接电子时代的挑战。

网络和信息技术的应用已经对人类社会、政治、文化和经济的正常发展形成巨大冲击,只有及早整合各个方面的资源,促进国际分工,才能更好地配合全球经济的运行。我国应该充分利用各种手段,充分意识到在未来网络和信息的重要作用,并提供相关社会培训,以促进国家转变观念,提高国民整体应用信息技术能力。

2.大力发展信息基础设施。

一个国家的电子商务发展依赖于一个完整的信息基础设施。电子商务在欧洲的实施和发展同样离不开信息基础设施建设。我们国家虽然在信息基础设施投资和建设中取得一些成功,形成基本的基础设施网络,但与其他国家地区相比,明显缺少对信息基础设施的投资,基础薄弱且没有足够的重视,离电子金融和电子商务高端化的目标差距较大。“信息贫困化”被认为是接下来几十年发展中国家经济发展的瓶颈,我们的信息基础设施的建设要为电子商务的发展提供后勤保障,日益缩小与发达国家之间的差距,促进电子商务的发展,加快信息基础设施建设,有效地利用网络资源。

3.专注创新国际贸易发展的形式。

电子商务正在改变国际贸易的内容和方式,对国际贸易和跨国公司提出新的要求和严峻的挑战,旧的规则已经不适应新时代对的变化,所以必须专注创新国际贸易发展的形式,其中包括安全性国际贸易规则及传统理论,自由贸易及网络税收,知识产权保护及电子合同诉讼等。新国际贸易在电子商务环境中应注重理论创新、宏观创新、创新政策和运行机制创新、业务创新、营销创新等新形式。

4.出台一套完整的规章制度政策。

中国目前的贸易法根据传统纸质交易制定,电子商务在新形势下的发展与传统贸易有诸多不同之处,所以大多数的规定对其并不适用,会产生很多难以逾越的障碍。为了确保相关的网络管理信息的安全和更好地对知识产权等问题进行保护,应加快对现行法律变革的步伐,建立并不断完善相关的针对电子商务时代的政策法规和规章。

5.加强专业人员的培训。

与电子商务和国际贸易相关的职业对从业人员的综合素质要求要比其他行业更严格,不仅需要在专业方面技能过硬,还需要熟悉计算机和网络信息技术知识,熟悉交易国的社会人文风俗、政治情况和法规商业现状。在当前情况下,只有从业人员的技能和素质过硬,依托电子商务的国际贸易发展的道路才能更通畅。结合国际贸易大市场的实际情况,我国高校要加大电子商务专业学生的培训力度,为未来培养一批可信赖的、有保障的后力军,以促进我国经济的发展。要建立企业与大学的长期合作关系,不断研究和学习实践中出现的新问题。电子商务的发展给国际贸易带来的新机会的同时也是一个挑战,信息技术桥梁不仅对国际贸易的影响很大,而且与经济和国家社会发展的紧密联系。因此,我们要以电子商务作为入口点,抓住电子商务发展机遇,把我国国际贸易的发展推上一个新台阶。

四、结语

电子商务时代下全球经济贸易向前加速发展,所以我们要不断自我革新,适应新形势下的新变化,适应新的复杂的国际贸易环境和方式。我国已经发展成为世界上最大的发展中国家,面临着复兴中华民族,人民全面走向幸福生活的重任,要不断接受挑战,抓住机遇,更好地适应新形势电子商务时代下的发展,获取更大的利益。我们要高度重视电子商务在国际经济与贸易发展中的作用,学习先进经验,取长补短,不断提高我国企业在国际中的竞争力,使我国整体的对外经济贸易水平走上一个新的台阶。