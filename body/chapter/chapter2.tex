\chapter{电子商务与国际贸易的基本介绍}
\section{电子商务的基本概念}
电子商务,望文生义,即通过电子设备进行的商务贸易往来,广义地说,就是一切与电子、数字化处理相关的商务活动都可以认为是电子商务,重点在于商务活动发生于网络计算机环境;狭义地说,就是通过网络手段进行价值产品的买卖商务活动,比如网络购物\cite{liuzantong2013}。

电子商务的发展方便了很多人,解决了时间、空间的冲突、限制,人们可以在网上进行购买,人们可以利用碎片化的时间享受购物过程,如今利用淘宝、京东、拼多多的购物平台的消费者越来越多。并且随着移动设备的迅速发展,人们只携带一个手机就可以通过任意平台进行贸易往来,甚至一些社交网站也提供购物平台的入口,可见电子商务发展之迅速\cite{yingxiaofan2014}。

\section{电子商务的主要特点}
电子商务具备以下特点:
\begin{enumerate}[(1)]
\setlength{\itemsep}{0ex}
\item 合作性高,电子商务的整个流程可以看作是一个协调工作的过程,生产、运营、销售等都是不可或缺的环节,线上线下的支付、货物的运送等也都离不开银行、物流部门、通信中心、技术服务人员等相互默契的合作,协调完成,各环节都需要精密合作。
\item 通用性强,电子商务作为目前广受欢迎的经营方式,具有通用性,数字化的网络世界中的交互都是基于互联网的操作。
\item 高效快速,在电子商务环境下,互联网将整个地球纳入地球村,人们的工作已经不再受到时间、地点的限制,客户可以以非常简单的方式在短时间内完成以往复杂的业务活动,并且能够随时随地通过互联网查询交易、资金、等信息。
\item 便捷严谨,通过规范贸易规则,基于互联网平台发布交易模板,能够有效地规范交易流程、合约等,交易双方仅通过简单地填写模板就可以完成贸易,
而模板既可以简化交易,又可以排除人为地漏洞,提高系统的严谨性\cite{zhouxiaoxian2014}。
\item 安全手段多,任何时候,安全性是交易的重中之重,相比与传统手段,电子贸易由于是基于互联网的交易,
拥有各种安全系数高的保密手段,比如加密签名、权限控制等,但是同时由于互联网的高效便捷,一旦出现安全漏洞,损失也比较巨大,因此安全性需要作为电子贸易的第一考量。
\end{enumerate}

\section{国际贸易的基本概念}
国际贸易又称通商,是指跨越国境的货品和服务交易,也就是国与国之间的贸易往来,能够调节国内的生产要素利用率,改善国际之间的供求关系,调整经济结构,增加财政收入。国际可以发生在两个或多个国之间,一起完成一项贸易,国与国之间的物品、价值商品交换,进行经济交流,跨境贸易,一般的国贸可以分为加工贸易、商品贸易、服务贸易等。根据贸易量的参与国,可以分为双边贸易、多边贸易,双边贸易指两个国家之间的贸易,一国进口一国出口,无需其他多家参与,多边贸易指三个或三个以上的国家参与贸易,在现代全球化经济体制下,多边贸易的发生更为频繁。
\section{国际贸易的主要特点}
国际贸易本质上也是商品、货币等价值的交换,但是国际贸易与老百姓日常接触的贸易存在很大区别,比如交易范围更广,距离更远,涉及领域更多等等。总的来看,国际贸易主要有以下几个特点:
\vspace{-0.5ex}
\begin{enumerate}[(1)]
\setlength{\itemsep}{0ex}
\item 波动性大,国际贸易牵扯两国或多国经济,但国与国之间不仅仅是经济合作,还有国际形势影响,因此,容易受到外部政治环境、双边关系等影响,波动性大。
\item 条款量大,国际贸易的参与国可能采用不同的语言,同时,英语作为世界的通用语言也经常要参与条款内容,合同的签订等还需要考虑双方的宗教、习俗、政策、保险、仲裁协商、收发货等方方面面。
\item 体量巨大,国际贸易由于是跨境交易,参与双方运输等成本较高,提高了准入门槛,贸易量也因此上升,单次交易一般都是大宗交易,即便是普通消费网购,一般也由网络平台进行统一综合处理发货。
\end{enumerate}
\section{国际物流的基本概念}
以现代物流理念为理论依据,在全球范围内以国为划分单位,进行工作物流分工,建设物流设施,应用物流信息传递技术,提高物流工作效率,最大化提升物流速度。互联网解决了国际贸易的商品交易、货币支付等上层问题,而基层的货物运送则完成了国际贸易的最后一环,完善了整个电子商务的闭环,国际贸易得以展开,从而飞速发展。随着全球经济一体化的进步,市场经济愈加需要国际贸易的支持,国际物流的支撑与保障必不可少。
\section{国际物流的主要特点}
现代国际物流目前展现出以下几个特点:
\begin{enumerate}[(1)]
\setlength{\itemsep}{0ex}
\item 距离远成本高,国际贸易的货物运送可想而知,必然伴随着跨境运输,运送距离可达几千上万公里,路途遥远,势必要采用高效管理的方式保证长途运输的稳定。同时由于距离的原因,导致管理、运输等物流成本巨大。
\item 巨头化,跨境物流成本高,准入门槛也就相应提升,从而出现几家物流巨头相互竞争的趋势,外界资本也流会入各大巨头而不是另起炉灶。
\item 外包化,物流成本的巨额费让许多企业放弃自己运输货物的方式,将物流外包给专业的物流公司,转而将更多的资金投入研发或者增值服务。
\end{enumerate}

\section{本章小结}
本章主要介绍了电子商务、国际贸易的基本概念以及两者的主要特点,同时简单介绍了在国际贸易中扮演重要角色的国际物流的概念及特点,现代国际物流可以说是国际贸易的桥梁,补全了电子商务的重要环节,增强了市场活性,完善了电子商务的贸易模式,推动了国际贸易的发展。
