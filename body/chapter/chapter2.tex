\chapter{现代物流对国际贸易的促进作用及研究}
所谓现代国际物流,实际上就是依据现代物流理念,在全球范围内,按照国际分工协作原则,实现的物流网络、物流设施以及物流信息技术的应用。现代物流业的发展,确实可以使得货物在全球范围内流动,实现了全球贸易的开展。并且随着全球经济一体化的进步,市场经济更是需要国际贸易的支持。因此,改变传统意义上的物流行业现状,对物流实现革命性变革,才是未来国际贸易发展过程中必不可少的支撑与保障。

1 现代物流的特点

针对现代物流发展现状分析,现代物流普遍呈现出以下几个特点:

其一,物流市场更为开放,且市场竞争日益激烈。许多外资企业为了增强自身在中国市场的占有率,便借助自己的实力和较为雄厚的资金支持,采取合资或者并购等方式,实行业务扩张,这将对我国物流业的发展带来冲击与影响。

其二,物流企业呈现出迅速增长的态势。大企业逐渐开始做强做大,而小企业也开始将自己的业务细分,尽可能做到精细。所以整个市场在定位上更加明确,社会中也陆续出现了一些具有较好口碑的专业物流企业,为我国物流行业的进步与发展带来了帮助。

其三,企业物流逐渐呈现分离与外包的趋势。一些物流企业的运营观念逐渐发生变化,其不再局限于最初的运输和仓储,而是向着增值服务方向发展。这就使得我国物流行业和制造业之间逐渐呈现出融合趋势,这对我国现代物流企业而言,既是机遇又是挑战。

2 国际贸易与现代物流的关系

从目前全球范围内的贸易发展状况来看,国际贸易与现代物流之间存在着较为紧密的关系。国际物流是现代国际贸易的工具与桥梁,所以往往可以打破国界,并且在最大程度上降低国际物流的成本。这原因主要是国际贸易作为国际物流生存的前提和基础,必须明确自身的发展速度决定了国际物流的发展速度。同时国际物流逐渐实现科学化实际上也可以为国际贸易提供保障,这一点尤其在跨国公司的不断扩展以及贸易往来过程中最为明显。所以现代物流业自身的服务深度越来越广,流程也逐渐增长,覆盖范围的日益扩大,使得国际贸易逐渐从生产、供应、营销、运输等多个方面提升了效益,进而推动了世界贸易的全面进步与发展。

3 现代物流对我国对外贸易的促进作用

3.1 降低了经营成本

现代物流行业的发展,对我国对外贸易活动的开展具有极强的促进作用。最直接地体现在降低了经营的成本。一般而言,要想提升贸易效益,首先应该综合考虑各国在经济方面所具有的优势。而优势则体现在对成本的控制与比较方面。但针对目前国际贸易发展现状分析,经济的高速发展使得成本降低的空间越来越小,可是在物流方面可以压缩的空间却越来越大。相关数据表明,我国物流行业及其相关行业一年的总支出大约为19000亿元,而物流经济效益在我国GDP总量中所占比重高达25%。故而,通过物流活动中各个环节的协调,可以使物流行业的经营成本不断降低,从而降低国际贸易成本,提升国际贸易效益与水平。

3.2 扩大了贸易规模

现代物流的发展也使得国际贸易的规模不断扩大,因为目前我国的对外贸易伙伴处于相对集中的状态,且大多为欧亚以及美洲国家。这足以看出我国对国际贸易的依赖程度相当高,不利于将市场扩展。而现代物流的发展则对国际贸易的规模有重要影响,尤其是现代物流发展中,口岸物流的建设以及交通运输网络的完善、电子信息技术的进步等,都为我国国际贸易活动的开展以及货品的运输提供了条件,从而扩大了我国国际贸易的整体规模。

3.3 加快市场反应速度的同时也提升了整个市场竞争力

我国市场中的商品普遍存在技术含量较低、劳动力投入较大的特点。也正是因为我国这种传统的产品生产形式,我国的产品对外竞争力较弱。为了有效改善这种现象,不仅需要增加产品的研发力度,另外还需要通过科学有效的管理,加快进出口企业对市场的反应速度。既需要降低交易成本,还需要生产适销对路、技术含量较高的产品,实现我国进出口经济的协调发展,从而提供较好的产销服务,提升我国产品在国际贸易竞争中的竞争实力。

探究现代物流对国际贸易的促进作用,实际上也是实现我国现代物流发展、促进我国国际贸易进步的重要举措。要想转变为贸易大国,现代物流企业在其中的重要性不言而喻。不仅需要从多方面进行改变,同时还需要历经长时间的付出。作为现代物流企业,要明确与国际贸易之间彼此的促进作用,这样将有利于实现现代物流对国际贸易的推动作用。

4 我国现代物流行业发展的对策

4.1 转变观念与意识,提高服务质量

作为现代物流企业,必须对传统的物流服务方式与手段进行创新。物流企业因为规模较小,所以往往缺乏相应的竞争实力。故而需要我国物流企业的领导人员转变自身的意识与观念,从而不断提升自身企业的服务质量。既需要为货主提供高效率的物流服务,还需要具有个性化服务特点,满足客户个性化的消费需要与服务需要。

4.2 加快现代物流基础设施建设

目前我国物流基础设施建设还相对落后,与国际贸易中需要实现的物流机械化、装备现代化要求较远。所以要想更好地发展我国现代物流行业,并且使之与国际贸易发展相契合,加大物流基础设施建设刻不容缓。比如可以建设集装箱专用码头,这是现代物流行业发展的基础。同时更应该满足国际物流连续性作业的要求,引进众多国外先进技术,实现物流设备的引进、研制、发展,从而提高货物的运输效率与水平。

4.3 培养专业物流人才

培养专业物流人才,是我国现代物流行业在未来的发展过程当中重要的基础和保障。所以许多院校也可以开设物流相关课程,其目的就是为物流行业提供更多既具有较强理论性,又具有极强操作性的专业人才。不仅需要大力引进国内外优秀人才,还需要建立起相应的培训与奖励制度,通过众多的方式留住优秀人才,进而实现我国现代物流企业的长久进步,为我国国际贸易的发展带来基础性帮助。

4.4 政府和国家给予足够的扶持与支持

政府和国家的支持,是现代物流企业发展的重要保障。尤其我国现代物流行业发展还处于初步发展阶段,所以往往需要国家的支持,这样才能规范我国的贸易市场,实行严格的进出口原则,从而让现代物流企业为国际贸易发展服务。

5 结 论

现代物流行业在未来的发展中,必将呈现出蓬勃发展的态势,同时其自身对国际贸易的影响也将越来越大。因此,现代物流企业必须通过合理的方式对自身进行创新与完善,制定出与国际贸易发展和市场经济相适应的对策,从而为世界贸易的繁荣与进步提供基础与条件。