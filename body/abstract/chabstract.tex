\chapter*{\markboth{摘要}{摘要}摘~要}	
在制导领域中,由于飞行器的飞行速度大幅度提升,随之而来的高速流场会产生严重的气动光学效应,导致目标图像偏移、模糊和抖动等现象,从而大大影响制导精度以及成像质量。而高科技精确打击武器对于目标追踪的精度要求越来越高,因此对于高速飞行器周围湍流流场的运动机理研究以及对流场中光传输特性研究的需求越来越迫切。

%由于高科技精确打击武器对于目标制导、追踪的要求越来越高,而飞行器的飞行速度正在大幅度提升,随之而来的气动光学效应导致的图像偏移、模糊和抖动等现象则会大大影响成像质量及制导精度,因此对于高速飞行器周围湍流流场的运动机理研究以及对流场中光传输特性研究的需求越来越迫切。

本文首先介绍了高速流场中气动光学效应的研究背景及意义,回顾了气动光学效应在理论及实验模拟上的研究发展历程,并对前人的工作进行了相应的总结。研究了高速以及超高音速流场的运动机理,分析了大涡模拟和雷诺平均两种主要的流场数值模拟方法,采用$k-\varepsilon$以及$k-\omega$计算模型基于混合算法推导出了剪切压力传输模型,即$k-\omega$修正二方程模型。采用ICEM建模软件建立了三维导弹及机载凸台模型,基于离散控制方程及定解条件的设置,使用Fluent求解器对不同速度下导弹和凸台周围的流场进行了数值模拟及结果分析。基于光传输理论推导了几种不同的气动光学评价指标的计算方法,并采用光线追迹及统计光学的研究方法对时间平均流场和湍流随机脉动流场的气动光学效应进行了分析。采用Gladstone-Dale定律将流场数值模拟的结果转化为光学参数,基于Mathematica数值计算工具分析了光束沿不同角度穿过不同速度下的流场后产生的波面畸变和斯特列尔比,采用光学传递函数分析了高斯光束经过2倍音速下凸台光学窗口周围混合流场时光斑的畸变情况,发现光斑的光强明显削弱并伴随微小偏移,并且光场不再满足高斯分布。




\vspace{2em}{\bf\noindent 关键词:}~~气动光学,湍流,计算流体力学,二方程模型,光学传递函数,斯特列尔比